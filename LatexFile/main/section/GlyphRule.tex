\section{拼合}
\subsection{基本層}
定義在SGBasic.sty的底層拼合命令為:\par
{\textbackslash}SubGlyph\{聲母\}\{介音\}{\{}韻腹{\}}\{韻尾一\}\{韻尾二\}{\{}調型{\}}{\{}調類{\}}。\par
用不到的部分直接在花括號中間留空,以下說明各個部件的格式。\par
\begin{itemize}
	\item 聲母:同聲母部分
	\item 介音:直接寫i、u、y或留空。
	\item 韻腹:同韻母部分。
	\item 擴展韻尾:顯示為與韻母拼合的韻尾,一般是offglide形式的i, u, y。
	\item 韻尾二:顯示為與聲調耦合的韻尾。
	\item 調型、調類:同聲調部分。
\end{itemize}
以下是示例表,以聲母\Symbol{pq}、韻母\Symbol{oofu}為例。\par
\MyTable{拼合示例圖}{tab:Tones}{c |C|C|C|C |C|C|C|C |C|C|C|C}{
	韻末 & \multicolumn{4}{c|}{陰} & \multicolumn{4}{c|}{陽} & \multicolumn{4}{c|}{中}\\
	\cline{2-13}
	{} & 平 & 上 & 去 & 入 & 平 & 上 & 去 & 入 & 平 & 上 & 去 & 入 \\
	\hline
	無 & \SubGlyph{pq}{}{oofu}{}{}{p}{i} & \SubGlyph{pq}{}{oofu}{}{}{s}{i} & \SubGlyph{pq}{}{oofu}{}{}{q}{i} & \SubGlyph{pq}{}{oofu}{}{}{r}{i} & \SubGlyph{pq}{}{oofu}{}{}{p}{a} & \SubGlyph{pq}{}{oofu}{}{}{s}{a} & \SubGlyph{pq}{}{oofu}{}{}{q}{a} & \SubGlyph{pq}{}{oofu}{}{}{r}{a} & \SubGlyph{pq}{}{oofu}{}{}{p}{z} & \SubGlyph{pq}{}{oofu}{}{}{s}{z} & \SubGlyph{pq}{}{oofu}{}{}{q}{z} & \SubGlyph{pq}{}{oofu}{}{}{r}{z} \\
	\hline
	唇 & \SubGlyph{pq}{}{oofu}{}{m}{p}{i} & \SubGlyph{pq}{}{oofu}{}{m}{s}{i} & \SubGlyph{pq}{}{oofu}{}{m}{q}{i} & \SubGlyph{pq}{}{oofu}{}{m}{r}{i} & \SubGlyph{pq}{}{oofu}{}{m}{p}{a} & \SubGlyph{pq}{}{oofu}{}{m}{s}{a} & \SubGlyph{pq}{}{oofu}{}{m}{q}{a} & \SubGlyph{pq}{}{oofu}{}{m}{r}{a} & \SubGlyph{pq}{}{oofu}{}{m}{p}{z} & \SubGlyph{pq}{}{oofu}{}{m}{s}{z} & \SubGlyph{pq}{}{oofu}{}{m}{q}{z} & \SubGlyph{pq}{}{oofu}{}{m}{r}{z} \\
	\hline
	舌尖 & \SubGlyph{pq}{}{oofu}{}{n}{p}{i} & \SubGlyph{pq}{}{oofu}{}{n}{s}{i} & \SubGlyph{pq}{}{oofu}{}{n}{q}{i} & \SubGlyph{pq}{}{oofu}{}{n}{r}{i} & \SubGlyph{pq}{}{oofu}{}{n}{p}{a} & \SubGlyph{pq}{}{oofu}{}{n}{s}{a} & \SubGlyph{pq}{}{oofu}{}{n}{q}{a} & \SubGlyph{pq}{}{oofu}{}{n}{r}{a} & \SubGlyph{pq}{}{oofu}{}{n}{p}{z} & \SubGlyph{pq}{}{oofu}{}{n}{s}{z} & \SubGlyph{pq}{}{oofu}{}{n}{q}{z} & \SubGlyph{pq}{}{oofu}{}{n}{r}{z} \\
	\hline
	舌根 & \SubGlyph{pq}{}{oofu}{}{ng}{p}{i} & \SubGlyph{pq}{}{oofu}{}{ng}{s}{i} & \SubGlyph{pq}{}{oofu}{}{ng}{q}{i} & \SubGlyph{pq}{}{oofu}{}{ng}{r}{i} & \SubGlyph{pq}{}{oofu}{}{ng}{p}{a} & \SubGlyph{pq}{}{oofu}{}{ng}{s}{a} & \SubGlyph{pq}{}{oofu}{}{ng}{q}{a} & \SubGlyph{pq}{}{oofu}{}{ng}{r}{a} & \SubGlyph{pq}{}{oofu}{}{ng}{p}{z} & \SubGlyph{pq}{}{oofu}{}{ng}{s}{z} & \SubGlyph{pq}{}{oofu}{}{ng}{q}{z} & \SubGlyph{pq}{}{oofu}{}{ng}{r}{z} \\
	\hline
	鼻化 & \SubGlyph{pq}{}{oofu}{}{hn}{p}{i} & \SubGlyph{pq}{}{oofu}{}{hn}{s}{i} & \SubGlyph{pq}{}{oofu}{}{hn}{q}{i} & \SubGlyph{pq}{}{oofu}{}{hn}{r}{i} & \SubGlyph{pq}{}{oofu}{}{hn}{p}{a} & \SubGlyph{pq}{}{oofu}{}{hn}{s}{a} & \SubGlyph{pq}{}{oofu}{}{hn}{q}{a} & \SubGlyph{pq}{}{oofu}{}{hn}{r}{a} & \SubGlyph{pq}{}{oofu}{}{hn}{p}{z} & \SubGlyph{pq}{}{oofu}{}{hn}{s}{z} & \SubGlyph{pq}{}{oofu}{}{hn}{q}{z} & \SubGlyph{pq}{}{oofu}{}{hn}{r}{z} \\
}%%

\subsection{映射層}
底層命令過於繁瑣。例如,漢語拼音的「xióng」,需要歷經如下推導過程:
\begin{itemize}
	\item x對應聲母「\bpmf{cf}」,其主碼為cf;
	\item iong是一個不透明拼式,等效於「ü+eng」,應拆解為「\bpmf{y}」與「\bpmf{nceng}」,其主碼分別為y與nceng;
	\item 華語的二聲對應陽平,其調型碼為p,調類碼為a;
	\item 組合以上資訊,調用代碼為\detokenize{\SubGlyph{cf}{y}{nceng}{}{}{p}{a}},效果為\SubGlyph{cf}{y}{nceng}{}{}{p}{a}。
\end{itemize}
有鑑於此,package內提供了六種語言的範例及接口sty檔案SubGlyph.sty。只需要了解拼式的聲韻調分別是哪些,即可自動產生對應的通用拼合式注音,例如:\detokenize{\ToSubGlyph{Hanpin}{x}{iong}{2}}直接得到\ToSubGlyph{Hanpin}{x}{iong}{2}。\par
以下是六種語言的拼字示例。如有疏漏,敬請母語者不吝賜教。\par
\MyTable{Example: Pronunciation of Numbers}{tab:Supported}{c|C|C|C|C|C|C|C|C|C|C|C}{%
	Language & Zero & One & Two & Three & Four & Five & Six & Seven & Eight & Nine & Ten \\
	\hline
	Mandarin &%
		\ToSubGlyph{Hanpin}{l}{ing}{2} &%
		\ToSubGlyph{Hanpin}{}{yi}{1} &%
		\ToSubGlyph{Hanpin}{}{er}{4} &%
		\ToSubGlyph{Hanpin}{s}{an}{1} &%
		\ToSubGlyph{Hanpin}{s}{i}{4} &%
		\ToSubGlyph{Hanpin}{}{wu}{3} &%
		\ToSubGlyph{Hanpin}{l}{iu}{4} &%
		\ToSubGlyph{Hanpin}{q}{i}{1} &%
		\ToSubGlyph{Hanpin}{b}{a}{1} &%
		\ToSubGlyph{Hanpin}{j}{iu}{3} &%
		\ToSubGlyph{Hanpin}{sh}{i}{2} \\%
	\hline
	Taiwanese Colloquial &%
		\ToSubGlyph{Tailo}{l}{ing}{5} &%
		\ToSubGlyph{Tailo}{ts}{it}{8} &%
		\ToSubGlyph{Tailo}{j}{i}{7} &%
		\ToSubGlyph{Tailo}{s}{ann}{1} &%
		\ToSubGlyph{Tailo}{s}{i}{3} &%
		\ToSubGlyph{Tailo}{g}{oo}{7} &%
		\ToSubGlyph{Tailo}{l}{ak}{8} &%
		\ToSubGlyph{Tailo}{tsh}{it}{4} &%
		\ToSubGlyph{Tailo}{p}{eh}{4} &%
		\ToSubGlyph{Tailo}{k}{au}{2} &%
		\ToSubGlyph{Tailo}{ts}{ap}{8} \\%
	\cline{1-1}\cline{3-3}\cline{5-8}\cline{10-12}
	Taiwanese Literary &%
		{} &%
		\ToSubGlyph{Tailo}{}{it}{4} &%
		{} &%
		\ToSubGlyph{Tailo}{s}{am}{1} &%
		\ToSubGlyph{Tailo}{s}{u}{3} &%
		\ToSubGlyph{Tailo}{ng}{oo}{2} &%
		\ToSubGlyph{Tailo}{l}{iok}{8} &%
		{} &%
		\ToSubGlyph{Tailo}{p}{at}{4} &%
		\ToSubGlyph{Tailo}{k}{iu}{2} &%
		\ToSubGlyph{Tailo}{s}{ip}{8} \\%
	\hline
	Hakka (Xiian) &%
		\ToSubGlyph{HagpinXiian}{l}{ang}{2} &%
		\ToSubGlyph{HagpinXiian}{}{id}{5} &%
		\ToSubGlyph{HagpinXiian}{ng}{i}{4} &%
		\ToSubGlyph{HagpinXiian}{s}{am}{1} &%
		\ToSubGlyph{HagpinXiian}{x}{i}{4} &%
		\ToSubGlyph{HagpinXiian}{}{ng}{3} &%
		\ToSubGlyph{HagpinXiian}{l}{iug}{5} &%
		\ToSubGlyph{HagpinXiian}{q}{id}{5} &%
		\ToSubGlyph{HagpinXiian}{b}{ad}{5} &%
		\ToSubGlyph{HagpinXiian}{g}{iu}{3} &%
		\ToSubGlyph{HagpinXiian}{s}{iib}{6} \\%
	\hline
	Cantonese &%
		\ToSubGlyph{Jyutping}{l}{ing}{4} &%
		\ToSubGlyph{Jyutping}{j}{at}{7} &%
		\ToSubGlyph{Jyutping}{j}{i}{6} &%
		\ToSubGlyph{Jyutping}{s}{aam}{1} &%
		\ToSubGlyph{Jyutping}{s}{ei}{3} &%
		\ToSubGlyph{Jyutping}{}{ng}{5} &%
		\ToSubGlyph{Jyutping}{l}{uk}{9} &%
		\ToSubGlyph{Jyutping}{c}{at}{7} &%
		\ToSubGlyph{Jyutping}{b}{aat}{8} &%
		\ToSubGlyph{Jyutping}{g}{au}{2} &%
		\ToSubGlyph{Jyutping}{s}{ap}{9} \\%
	\hline
	Suzhounese &%
		\ToSubGlyph{WuphinSoutseu}{l}{in}{2} &%
		\ToSubGlyph{WuphinSoutseu}{}{iq}{7} &%
		\ToSubGlyph{WuphinSoutseu}{gn}{i}{6} &%
		\ToSubGlyph{WuphinSoutseu}{s}{e}{1} &%
		\ToSubGlyph{WuphinSoutseu}{s}{y}{5} &%
		\ToSubGlyph{WuphinSoutseu}{}{ng}{6} &%
		\ToSubGlyph{WuphinSoutseu}{l}{oq}{8} &%
		\ToSubGlyph{WuphinSoutseu}{tsh}{iq}{7} &%
		\ToSubGlyph{WuphinSoutseu}{p}{oq}{7} &%
		\ToSubGlyph{WuphinSoutseu}{c}{ieu}{3} &%
		\ToSubGlyph{WuphinSoutseu}{z}{eq}{8} \\%
	\hline
	Fuzhounese &%55=1, 33=2, 213=3, 24=4, 53=5, 242=7, 5=8
		\ToSubGlyph{Yngping}{l}{ing}{5} &%
		\ToSubGlyph{Yngping}{s}{uoh}{8} &%文讀eik4?
		\ToSubGlyph{Yngping}{n}{ei}{7} &%
		\ToSubGlyph{Yngping}{s}{ang}{1} &%
		\ToSubGlyph{Yngping}{s}{ei}{3} &%
		\ToSubGlyph{Yngping}{ng}{ou}{7} &%文讀ngu2?
		\ToSubGlyph{Yngping}{l}{eoyk}{8} &%文讀lyk8?
		\ToSubGlyph{Yngping}{c}{eik}{4} &%
		\ToSubGlyph{Yngping}{b}{aik}{4} &%
		\ToSubGlyph{Yngping}{g}{au}{2} &%文讀giu2?
		\ToSubGlyph{Yngping}{s}{eik}{8} \\%
}%%
作者也提供了python腳本,只需要整理該語言的羅馬字方案並參考寫好的檔案撰寫配置檔,即可自動產生對應的sty檔案。腳本會自動更新頂層的接口檔案SubGlyph.sty,不需要手動新增\detokenize{\usepackage{SGYoursystem.sty}}。為了防止撞名,建議命名方式為:\par
\begin{itemize}
	\item 若該拼音方案標準化程度高,請直接使用該語言的簡稱加上該語言「拼」或「羅」的讀音,例如「Jyutping(粵拼)」、「Tailo(台羅)」、「Hanpin(漢拼)」。
	\item 若該拼音方案屬於一個大類的子方案,請在後面加上地域名稱在該方案中的拼式,例如「WuphinSoutseu(吳拼蘇州)」、「HagpinXiian(客拼四縣)」。華語音節組合少,羅馬化後極易重名,所以若您的語言羅馬字方案與漢語拼音有所差異,請直接使用該方案書寫地域名稱。
\end{itemize}

\subsection{混合排版示例}
\begin{itemize}
	\item 華語:「和」讀音異常複雜,有\ToSubGlyph{Hanpin}{h}{an}{4}、\ToSubGlyph{Hanpin}{h}{e}{2}、\ToSubGlyph{Hanpin}{h}{uo}{4}、\ToSubGlyph{Hanpin}{h}{u}{2}、\ToSubGlyph{Hanpin}{h}{e}{4}等讀音。
	\item 台語:伊佇\ToSubGlyph{Tailo}{h}{ia}{1}毋知企偌久矣。
\end{itemize}

\clearpage