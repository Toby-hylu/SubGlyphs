\section*{前言}

漢語族語言(Sinitic languages)雖種類繁多,然迄今唯有國語/普通話具備官方地位與成熟、普及的書面語體系。其他語言如台語、客語、粵語等,雖具有準官方地位,並各自發展出一套約定俗成的書寫標準,然而仍然面臨一些問題。\par
採用「純漢字書寫」時,遇到的主要問題是「有音無字」的處理困境:\par
\begin{itemize}
	\item 造新字:粵語白話文傾向創造新字(如「嚟」),通常以口字旁加聲旁構形,雖然較能準確表音但字型可能複雜;
	\item 用華語音近字:台語白話文則常寫作字型簡單、音近的華語字(如「袂」),惟這一做法致使「不表音、不表意」的現象普遍存在;
	\item 用同音字:台語白話文也經常使用,如「魯」肉飯,但失去表意能力,例如曾經有外國人將其翻譯為「Shandong meat rice」(誤以為「魯」是山東的意思);
	\item 用本字:看似是最佳方案,但首先台語、粵語乃至吳語、閩東語等都存在底層非漢語詞,其次即使有本字也可能生僻難解,如表野草莓之「藨」字不認識者十之八九。
	\item 純拉丁書寫:雖然有一定社群,但此外,變調豐富的語言如台語、福州話或蘇州話等,拉丁字母無疑需要引入大量的連字符表示變調單位,
\end{itemize}
採用「純拉丁書寫」時,遇到的主要是方案設計困境:\par
\begin{itemize}
	\item 語言差異:拉丁字母擅長表達的音節與漢語族差異太大,這導致羅馬化方案一般都存在複雜的二合、三合乃至多合字母。例如,表達「咀嚼脆的食物」的擬聲詞以台羅方案書寫是難以認讀的「kha̍unnh」,包括二合字母kh、四合韻母aunn、入聲韻尾h等部分,而這個字往往以疊詞的形式出現。
	\item 連字符:連讀變調豐富的語言需要明確區分變調單位,這導致書寫時引入大量的連字符,可讀性與美觀大打折扣,還是以上一個例子,該擬聲詞往往是疊詞的形式「kha̍unnh-kha̍unnh」。
	\item 方案衝突:成熟的拼音方案往往有極大的影響力,迫使設計其他語言的方案時不得不配合已經成熟的方案,例如一些中國民間的拼音方案不得不使用bb或bh表達「濁雙唇塞音」,因為b已經被漢語拼音用來表達「清不送氣雙唇塞音」了。
\end{itemize}

漢字與羅馬字混寫雖然看似綜合了上述兩者的好處,但它在美學上缺點極其顯著:
\begin{itemize}
	\item 看起來極度不成熟,像是國小學生遇到不會寫的字,就寫成拼音代替那樣。
	\item 排版美感糟糕:拉丁文字與東亞文字各自發展出了成熟的排版美學,但這兩種美學本質上有衝突:拉丁文字讓每個字元有不同的寬度使得書面流暢靈動,東亞文字保證每個字元等寬使得書面整齊乾淨。混合排版時就會發生嚴重的不協調感,理工類書籍可以接受,但文學類書籍的排版美學也是其重要的組成部分。
\end{itemize}

既然混排看起來是有潛力的方向,我們可從已有成熟東亞文字混排歷史的日文與韓文的書寫策略獲得啟發:
\begin{itemize}
	\item 日文的策略在於:能書寫則用漢字,無合適漢字則改用假名標音,既保留語意,也兼顧語音,實用而靈活。
	\item 韓文的策略則體現在:設計出諺文體系,將音節結構壓縮於一個「漢字大小」的方格中,實現書寫單元的簡潔與規律。
\end{itemize}

有鑑於此,本文嘗試提出一種結合兩者優點的書寫設計:以注音符號為基礎,並參考韓文的音節壓縮方式,將單一音節(含擴展符號)封裝為一個方塊單位,進而配合類似日文的漢字-假名混排機制,達成語音準確與排版美觀的平衡。\par

在技術層面上,為避免新增 Unicode 碼位,本文提出的擴展策略採用既有字符結合方式表現濁音與鼻化:
\begin{itemize}
	\item 使用濁點(゛)與半濁點(゜)擴展符號表;
	\item 適當引入一些筆畫簡單的漢字或片假名字元;
\end{itemize}
目前,為兼容所有必要字符,本文使用支援完整注音符號集及片假名的字型TW-98楷體(及其增補字型)。展望未來,若本系統獲得應用與推廣,致使Unicode接納新注音符號字元,則期逐步汰換借用的日文假名與簡單漢字,構建更加純粹注音風格的注音合字系統。\par
\clearpage