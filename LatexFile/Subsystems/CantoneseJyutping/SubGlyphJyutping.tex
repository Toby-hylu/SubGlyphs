\documentclass[12pt]{article}

% Please typeset with XeLaTeX

\input{../../premable/of\_Subsystems.tex}

\begin{document}
% 文章內容

%自定義大標題
\begin{center}
    {\Huge \textbf{粵拼} \par}
    {\Huge \textbf{Jyutping} \par}
    {\Large Hylu \par}
    \hrulefill
\end{center}
This document provides SubGlyph of Jyutping, a romanization system of Cantonese. I'd like to thank the following websites for their comprehensive information. \par
\begin{itemize}
	\item 粵語審音配詞字庫:https://humanum.arts.cuhk.edu.hk/Lexis/lexi-can/
	\item 粵拼:https://jyutping.org
\end{itemize}

\section{Onsets}
Glides are treated as special onsets in most Romanization of Cantonese including Jyutping. Thus, I followed the tradition in the SubGlyph system of Cantonese as well. \par
{%
\setmainfont{DoulosSIL}%
\MyTable{Onsets}{tab:Onset}{L|L|L|L|L|L}{%
	Group & P & T & S & C & K \\
	\hline
	全清 & {/p/} & {/t/} & {/t͡s/} & {/ȶ͡ɕ/} & {/k/} \\
	{} & {\bpmf{\Onset{Tailo}{p}{}} p} & {\bpmf{\Onset{Tailo}{t}{}} t} & {\bpmf{\Onset{Tailo}{ts}{}} ts} & {\bpmf{\Onset{Tailo}{tsi}{}} tsi} & {\bpmf{\Onset{Tailo}{k}{}} k} \\ 
	\hline
	次清 & {/pʰ/} & {/tʰ/} & {/t͡sʰ/} & {/ȶ͡ɕʰ/} & {/kʰ/} \\
	{} & {\bpmf{\Onset{Tailo}{ph}{}} ph} & {\bpmf{\Onset{Tailo}{th}{}} th} & {\bpmf{\Onset{Tailo}{tsh}{}} tsh} & {\bpmf{\Onset{Tailo}{tshi}{}} tshi} & {\bpmf{\Onset{Tailo}{kh}{}} kh} \\
	\hline
	全濁 & {/b/} & {/l/} & {/d͡z/} & {ȡ͡ʑ} & {/g/} \\
	{} & {\bpmf{\Onset{Tailo}{b}{}} b} & {\bpmf{\Onset{Tailo}{l}{}} l} & {\bpmf{\Onset{Tailo}{j}{}} j} & {\bpmf{\Onset{Tailo}{ji}{}} ji} & {\bpmf{\Onset{Tailo}{g}{}} g} \\
	\hline
	次濁 & {/m/} & \multicolumn{2}{c|}{/n/} & {} & {/ŋ/} \\
	{} & {\bpmf{\Onset{Tailo}{m}{}} m} & \multicolumn{2}{c|}{\bpmf{\Onset{Tailo}{n}{}} n} & {} & {\bpmf{\Onset{Tailo}{ng}{}} ng} \\
	\hline
	清 & {} & {} & {/s/} & {/ɕ/} & {/h/} \\
	{} & {} & {} & {\bpmf{\Onset{Tailo}{s}{}} s} & {\bpmf{\Onset{Tailo}{si}{}} si} & {\bpmf{\Onset{Tailo}{h}{}} h} \\
}%
}%%

\section{Tones}
In SubGlyphs, coda p, t and k are treated as the checked tone form of m, n and ng. Please switch checked tone labeled as 1, 3 and 6 in Jyutping to 7, 8 and 9 manually while using The {\textbackslash}ToSubGlyphs command. \par
The 8th tone is lower yin checked tone. I used expanded checked tone to describe it. \par
\MyTable{Tones}{tab:Tone}{c|C|C|C|C|C|C|C|C|C}{%
	Number & 1 & 2 & 3 & 4 & 5 & 6 & 7 & 8 & 9 \\
	\hline
	Code & p,i & s,i & q,i & p,a & s,a & q,a & r,i & r,z & r,a \\
	\hline
	Basic glyph & \ToneCoda{\Tonetype{Jyutping}{1}}{\Toneclass{Jyutping}{1}}{}  & \ToneCoda{\Tonetype{Jyutping}{2}}{\Toneclass{Jyutping}{2}}{} & \ToneCoda{\Tonetype{Jyutping}{3}}{\Toneclass{Jyutping}{3}}{} & \ToneCoda{\Tonetype{Jyutping}{4}}{\Toneclass{Jyutping}{4}}{} & \ToneCoda{\Tonetype{Jyutping}{5}}{\Toneclass{Jyutping}{5}}{} & \ToneCoda{\Tonetype{Jyutping}{6}}{\Toneclass{Jyutping}{6}}{} & \ToneCoda{\Tonetype{Jyutping}{7}}{\Toneclass{Jyutping}{7}}{} & \ToneCoda{\Tonetype{Jyutping}{8}}{\Toneclass{Jyutping}{8}}{} & \ToneCoda{\Tonetype{Jyutping}{9}}{\Toneclass{Jyutping}{9}}{} \\
	\hline
	Labial glyph & \ToneCoda{\Tonetype{Jyutping}{1}}{\Toneclass{Jyutping}{1}}{m}  & \ToneCoda{\Tonetype{Jyutping}{2}}{\Toneclass{Jyutping}{2}}{m} & \ToneCoda{\Tonetype{Jyutping}{3}}{\Toneclass{Jyutping}{3}}{m} & \ToneCoda{\Tonetype{Jyutping}{4}}{\Toneclass{Jyutping}{4}}{m} & \ToneCoda{\Tonetype{Jyutping}{5}}{\Toneclass{Jyutping}{5}}{m} & \ToneCoda{\Tonetype{Jyutping}{6}}{\Toneclass{Jyutping}{6}}{m} & \ToneCoda{\Tonetype{Jyutping}{7}}{\Toneclass{Jyutping}{7}}{m} & \ToneCoda{\Tonetype{Jyutping}{8}}{\Toneclass{Jyutping}{8}}{m} & \ToneCoda{\Tonetype{Jyutping}{9}}{\Toneclass{Jyutping}{9}}{m} \\
	\hline
	Alveolar glyph & \ToneCoda{\Tonetype{Jyutping}{1}}{\Toneclass{Jyutping}{1}}{n}  & \ToneCoda{\Tonetype{Jyutping}{2}}{\Toneclass{Jyutping}{2}}{n} & \ToneCoda{\Tonetype{Jyutping}{3}}{\Toneclass{Jyutping}{3}}{n} & \ToneCoda{\Tonetype{Jyutping}{4}}{\Toneclass{Jyutping}{4}}{n} & \ToneCoda{\Tonetype{Jyutping}{5}}{\Toneclass{Jyutping}{5}}{n} & \ToneCoda{\Tonetype{Jyutping}{6}}{\Toneclass{Jyutping}{6}}{n} & \ToneCoda{\Tonetype{Jyutping}{7}}{\Toneclass{Jyutping}{7}}{n} & \ToneCoda{\Tonetype{Jyutping}{8}}{\Toneclass{Jyutping}{8}}{n} & \ToneCoda{\Tonetype{Jyutping}{9}}{\Toneclass{Jyutping}{9}}{n} \\
	\hline
	Labial glyph & \ToneCoda{\Tonetype{Jyutping}{1}}{\Toneclass{Jyutping}{1}}{ng}  & \ToneCoda{\Tonetype{Jyutping}{2}}{\Toneclass{Jyutping}{2}}{ng} & \ToneCoda{\Tonetype{Jyutping}{3}}{\Toneclass{Jyutping}{3}}{ng} & \ToneCoda{\Tonetype{Jyutping}{4}}{\Toneclass{Jyutping}{4}}{ng} & \ToneCoda{\Tonetype{Jyutping}{5}}{\Toneclass{Jyutping}{5}}{ng} & \ToneCoda{\Tonetype{Jyutping}{6}}{\Toneclass{Jyutping}{6}}{ng} & \ToneCoda{\Tonetype{Jyutping}{7}}{\Toneclass{Jyutping}{7}}{ng} & \ToneCoda{\Tonetype{Jyutping}{8}}{\Toneclass{Jyutping}{8}}{ng} & \ToneCoda{\Tonetype{Jyutping}{9}}{\Toneclass{Jyutping}{9}}{ng} \\
}%%

\section{Rimes}
I listed every possible rimes, even if it is not used in real speech. \par
{\setmainfont{DoulosSIL}
\MyTable{Rimes}{tab:Rimes}{CC| CC| CC| CC}{%
	\hline
	/ɿ,ʅ/ & y & \multicolumn{4}{c|}{\MaybeGray{}} & /ʮ,ʯ/ & yu \\
	\ToSubGlyph{WuphinSoutseu}{s}{y}{1} & {師} & \multicolumn{4}{c|}{\MaybeGray{}} & \ToSubGlyph{WuphinSoutseu}{ts}{yu}{1} & {知} \\
	\hline
	{} & {} & /ʑ̩/ & i & /u/ & u & /ʑ̩ʷ/ & iu \\
	{} & {} & \ToSubGlyph{WuphinSoutseu}{f}{i}{1} & {非} & \ToSubGlyph{WuphinSoutseu}{p}{u}{5} & {布} & \ToSubGlyph{WuphinSoutseu}{c}{iu}{1} & {居} \\
	\hline
	/ən/ & en & /in/ & in & /wən/ & uen & /yn/ & iun \\
	\ToSubGlyph{WuphinSoutseu}{p}{en}{3} & {本} & \ToSubGlyph{WuphinSoutseu}{b}{in}{2} & {平} & \ToSubGlyph{WuphinSoutseu}{kh}{uen}{1} & {昆} & \ToSubGlyph{WuphinSoutseu}{c}{iun}{1} & {軍} \\
	\hline
	/əʔ/ & eq & /jəʔ/ & iq & /wəʔ/ & ueq & /ɥəʔ/ & iuq \\
	\ToSubGlyph{WuphinSoutseu}{t}{eq}{7} & {得} & \ToSubGlyph{WuphinSoutseu}{c}{iq}{7} & {吉} & \ToSubGlyph{WuphinSoutseu}{k}{ueq}{7} & {骨} & \ToSubGlyph{WuphinSoutseu}{y}{uq}{8} & {月} \\
	\hline
	/ᴇ/ & e & /i/ & ie & /wᴇ/ & ue & {} & {} \\
	\ToSubGlyph{WuphinSoutseu}{p}{e}{1} & {班} & \ToSubGlyph{WuphinSoutseu}{th}{ie}{1} & {天} & \ToSubGlyph{WuphinSoutseu}{h}{ue}{1} & {灰} & {} & {} \\
	\hline
	/ã/ & an & /jã/ & ian & /wã/ & uan & {} & {} \\ 
	\ToSubGlyph{WuphinSoutseu}{t}{an}{3} & {打} & \ToSubGlyph{WuphinSoutseu}{}{ian}{1} & {央} & \ToSubGlyph{WuphinSoutseu}{}{uan}{1} & {橫} & {} & {} \\
	\hline
	/aʔ/ & aeq & /jaʔ/ & iaeq & /waʔ/ & uaeq & /ɥaʔ/ & iuaeq \\
	\ToSubGlyph{WuphinSoutseu}{}{aeq}{7} & {鴨} & \ToSubGlyph{WuphinSoutseu}{c}{iaeq}{7} & {甲} & \ToSubGlyph{WuphinSoutseu}{w}{aeq}{8} & {滑} & \ToSubGlyph{WuphinSoutseu}{y}{uaeq}{8} & {曰} \\
	\hline
	/ø/ & oe & /jø/ & ioe & /wø/ & uoe & {} & {} \\
	\ToSubGlyph{WuphinSoutseu}{p}{oe}{5} & {半} & \ToSubGlyph{WuphinSoutseu}{gn}{ioe}{2} & {原} & \ToSubGlyph{WuphinSoutseu}{k}{uoe}{1} & {官} & {} & {} \\
	\hline
	/o/ & o & /jo/ & io & {} & {} & {} & {} \\
	\ToSubGlyph{WuphinSoutseu}{p}{o}{1} & {巴} & \ToSubGlyph{WuphinSoutseu}{}{io}{1} & {亞} & {} & {} & {} & {} \\
	\hline
	/oŋ/ & on & /joŋ/ & ion & {} & {} & {} & {} \\
	\ToSubGlyph{WuphinSoutseu}{k}{on}{1} & {工} & \ToSubGlyph{WuphinSoutseu}{gn}{ion}{2} & {濃} & {} & {} & {} & {} \\
	\hline
	/oʔ/ & oq & /joʔ/ & ioq & {} & {} & {} & {} \\
	\ToSubGlyph{WuphinSoutseu}{p}{oq}{7} & {八} & \ToSubGlyph{WuphinSoutseu}{ch}{ioq}{7} & {曲} & {} & {} & {} & {} \\
	\hline
	/ɑ/ & a & /jɑ/ & ia & /wɑ/ & ua & {} & {} \\
	\ToSubGlyph{WuphinSoutseu}{h}{a}{3} & {蟹} & \ToSubGlyph{WuphinSoutseu}{s}{ia}{3} & {寫} & \ToSubGlyph{WuphinSoutseu}{kh}{ua}{5} & {快} & {} & {} \\
	\hline
	/ɑ̃/ & aon & /jɑ̃/ & iaon & /wɑ̃/ & uaon & {} & {} \\
	\ToSubGlyph{WuphinSoutseu}{p}{aon}{1} & {幫} & \ToSubGlyph{WuphinSoutseu}{c}{iaon}{1} & {江} & \ToSubGlyph{WuphinSoutseu}{k}{uaon}{1} & {光} & {} & {} \\
	\hline
	/ɑʔ/ & aq & /jɑʔ/ & iaq & {} & {} & {} & {} \\
	\ToSubGlyph{WuphinSoutseu}{kh}{aq}{7} & {客} & \ToSubGlyph{WuphinSoutseu}{y}{aq}{8} & {俠} & {} & {} & {} & {} \\
	\hline
	/øʏ̆/ & eu & /y/ & ieu & {} & {} & {} & {} \\
	\ToSubGlyph{WuphinSoutseu}{ts}{eu}{1} & {州} & \ToSubGlyph{WuphinSoutseu}{gn}{ieu}{2} & {牛} & {} & {} & {} & {} \\
	\hline
	/æ/ & au & /jæ/ & iau & {} & {} & {} & {} \\
	\ToSubGlyph{WuphinSoutseu}{p}{au}{1} & {包} & \ToSubGlyph{WuphinSoutseu}{t}{iau}{1} & {刁} & {} & {} & {} & {} \\
	\hline
	/əu̯/ & ou & {} & {} & {} & {} & {} & {} \\
	\ToSubGlyph{WuphinSoutseu}{th}{ou}{3} & {土} & {} & {} & {} & {} & {} & {} \\
	\hline
	/l̩/ & er & /n̩/ & n & /ŋ̍/ & ng & /m̩/ & m \\
	\ToSubGlyph{WuphinSoutseu}{}{er}{6} & {而} & \ToSubGlyph{WuphinSoutseu}{}{n}{6} & {唔} & \ToSubGlyph{WuphinSoutseu}{}{ng}{6} & {五} & \ToSubGlyph{WuphinSoutseu}{}{m}{2} & {嘸} \\
}%
}%%

\end{document}