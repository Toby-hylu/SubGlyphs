\documentclass[12pt]{article}

% Please typeset with XeLaTeX

\input{../../premable/of\_Subsystems.tex}

\begin{document}
% 文章內容

%自定義大標題
\begin{center}
    {\Huge \textbf{臺語羅馬字} \par}
    {\Huge \textbf{Taiwanese Tâi-lô} \par}
    {\Large Hylu \par}
    \hrulefill
\end{center}

This document provides SubGlyph of Tâi-Lô, a romanization system of Taiwanese. I'd like to thank the following websites for their comprehensive information. \par
\begin{itemize}
	\item 教育部臺灣台語常用詞辭典:https://sutian.moe.edu.tw/zh-hant/
	\item 臺灣台語羅馬字拼音教學網:https://tailo.moe.edu.tw
\end{itemize}

\section{Onset}
{\setmainfont{DoulosSIL}%
\MyTable{Onsets}{tab:Onset}{c|C|C|C|C|C}{%
	Type & P & T & S & C & K \\
	\hline
	全清 & /p/ & /t/ & /t͡s/ & /ȶ͡ɕ/ & /k/ \\
	{} & {\bpmf{\Onset{HagpinXiian}{b}{}} b} & {\bpmf{\Onset{HagpinXiian}{d}{}} d} & {\bpmf{\Onset{HagpinXiian}{z}{}} z} & {\bpmf{\Onset{HagpinXiian}{j}{}} j} & {\bpmf{\Onset{HagpinXiian}{g}{}} g} \\
	\hline
	次清 & /pʰ/ & /tʰ/ & /t͡sʰ/ & /ȶ͡ɕʰ/ & /kʰ/ \\
	{} & {\bpmf{\Onset{HagpinXiian}{p}{}} p} & {\bpmf{\Onset{HagpinXiian}{t}{}} t} & {\bpmf{\Onset{HagpinXiian}{c}{}} c} & {\bpmf{\Onset{HagpinXiian}{q}{}} q} & {\bpmf{\Onset{HagpinXiian}{k}{}} k} \\
	\hline
	次濁 & /m/ & /n/ & {} & {} & /ŋ/ \\
	{} & {\bpmf{\Onset{HagpinXiian}{m}{}} m} & {\bpmf{\Onset{HagpinXiian}{n}{}} n} & {} & {} & {\bpmf{\Onset{HagpinXiian}{ng}{}} ng} \\
	\hline
	清 & /f/ & {} & /s/ & /ɕ/ & /h/ \\
	{} & {\bpmf{\Onset{HagpinXiian}{f}{}} f} & {} & {\bpmf{\Onset{HagpinXiian}{s}{}} s} & {\bpmf{\Onset{HagpinXiian}{x}{}} x} & {\bpmf{\Onset{HagpinXiian}{h}{}} h} \\
	\hline
	濁 & /v/ & /l/ & {} & {} & {} \\
	{} & {\bpmf{\Onset{HagpinXiian}{v}{}} v} & {\bpmf{\Onset{HagpinXiian}{l}{}} l} & {} & {} & {} \\
}%
}%%
Note: C-group is actually the palatalized form of S-group. \par

\clearpage

\section{Tone}
\MyTable{Tones}{tab:Tone}{c|C|C|C|C|C|C|C|C|C}{%
	Number & 1 & 2 & 3 & 4 & 5 & 6 & 7 & 8 & 9 \\
	\hline
	Code & p,i & s,i & q,i & p,a & s,a & q,a & r,i & r,z & r,a \\
	\hline
	Basic glyph & \ToneCoda{\Tonetype{Jyutping}{1}}{\Toneclass{Jyutping}{1}}{}  & \ToneCoda{\Tonetype{Jyutping}{2}}{\Toneclass{Jyutping}{2}}{} & \ToneCoda{\Tonetype{Jyutping}{3}}{\Toneclass{Jyutping}{3}}{} & \ToneCoda{\Tonetype{Jyutping}{4}}{\Toneclass{Jyutping}{4}}{} & \ToneCoda{\Tonetype{Jyutping}{5}}{\Toneclass{Jyutping}{5}}{} & \ToneCoda{\Tonetype{Jyutping}{6}}{\Toneclass{Jyutping}{6}}{} & \ToneCoda{\Tonetype{Jyutping}{7}}{\Toneclass{Jyutping}{7}}{} & \ToneCoda{\Tonetype{Jyutping}{8}}{\Toneclass{Jyutping}{8}}{} & \ToneCoda{\Tonetype{Jyutping}{9}}{\Toneclass{Jyutping}{9}}{} \\
	\hline
	Labial glyph & \ToneCoda{\Tonetype{Jyutping}{1}}{\Toneclass{Jyutping}{1}}{m}  & \ToneCoda{\Tonetype{Jyutping}{2}}{\Toneclass{Jyutping}{2}}{m} & \ToneCoda{\Tonetype{Jyutping}{3}}{\Toneclass{Jyutping}{3}}{m} & \ToneCoda{\Tonetype{Jyutping}{4}}{\Toneclass{Jyutping}{4}}{m} & \ToneCoda{\Tonetype{Jyutping}{5}}{\Toneclass{Jyutping}{5}}{m} & \ToneCoda{\Tonetype{Jyutping}{6}}{\Toneclass{Jyutping}{6}}{m} & \ToneCoda{\Tonetype{Jyutping}{7}}{\Toneclass{Jyutping}{7}}{m} & \ToneCoda{\Tonetype{Jyutping}{8}}{\Toneclass{Jyutping}{8}}{m} & \ToneCoda{\Tonetype{Jyutping}{9}}{\Toneclass{Jyutping}{9}}{m} \\
	\hline
	Alveolar glyph & \ToneCoda{\Tonetype{Jyutping}{1}}{\Toneclass{Jyutping}{1}}{n}  & \ToneCoda{\Tonetype{Jyutping}{2}}{\Toneclass{Jyutping}{2}}{n} & \ToneCoda{\Tonetype{Jyutping}{3}}{\Toneclass{Jyutping}{3}}{n} & \ToneCoda{\Tonetype{Jyutping}{4}}{\Toneclass{Jyutping}{4}}{n} & \ToneCoda{\Tonetype{Jyutping}{5}}{\Toneclass{Jyutping}{5}}{n} & \ToneCoda{\Tonetype{Jyutping}{6}}{\Toneclass{Jyutping}{6}}{n} & \ToneCoda{\Tonetype{Jyutping}{7}}{\Toneclass{Jyutping}{7}}{n} & \ToneCoda{\Tonetype{Jyutping}{8}}{\Toneclass{Jyutping}{8}}{n} & \ToneCoda{\Tonetype{Jyutping}{9}}{\Toneclass{Jyutping}{9}}{n} \\
	\hline
	Labial glyph & \ToneCoda{\Tonetype{Jyutping}{1}}{\Toneclass{Jyutping}{1}}{ng}  & \ToneCoda{\Tonetype{Jyutping}{2}}{\Toneclass{Jyutping}{2}}{ng} & \ToneCoda{\Tonetype{Jyutping}{3}}{\Toneclass{Jyutping}{3}}{ng} & \ToneCoda{\Tonetype{Jyutping}{4}}{\Toneclass{Jyutping}{4}}{ng} & \ToneCoda{\Tonetype{Jyutping}{5}}{\Toneclass{Jyutping}{5}}{ng} & \ToneCoda{\Tonetype{Jyutping}{6}}{\Toneclass{Jyutping}{6}}{ng} & \ToneCoda{\Tonetype{Jyutping}{7}}{\Toneclass{Jyutping}{7}}{ng} & \ToneCoda{\Tonetype{Jyutping}{8}}{\Toneclass{Jyutping}{8}}{ng} & \ToneCoda{\Tonetype{Jyutping}{9}}{\Toneclass{Jyutping}{9}}{ng} \\
}%%

\section{Rimes}
{%
\setmainfont{DoulosSIL}%
\setlength{\tabcolsep}{1pt}%
\MyTable{Rimes with Canonical Consonantal Coda}{tab:LitRime}{CC|CC|cC|cC|cC|CC}{
	/-m/ & -m & /-p̚/  & -p & /-n/ & -n & /-t̚/ & -t & /-ŋ/ & -ng & /-k̚/ & -k \\
	\hline
	/im/ & im & /ip̚/ & ip & /in/ & in & /ɪt̚/ & it & /iŋ/ & ing & /ɪk̚/ & ik \\
	{\ToSubGlyph{Tailo}{kh}{im}{5}} & {琴} & {\ToSubGlyph{Tailo}{ji}{ip}{8}} & {入} & {\ToSubGlyph{Tailo}{k}{in}{2}} & {緊} & {\ToSubGlyph{Tailo}{t}{it}{8}} & {直} & {\ToSubGlyph{Tailo}{b}{ing}{5}} & {明} & {\ToSubGlyph{Tailo}{si}{ik}{4}} & {色} \\
	\hline
	/ɔm/ & om & /ɔp̚/ & op & /un/ & un & /ut̚/ & ut & /ɔŋ/ & ong & /ɔk̚/ & ok \\
	{\ToSubGlyph{Tailo}{s}{om}{1}} & {蔘} & {\ToSubGlyph{Tailo}{l}{op}{8}} & {槖} & {\ToSubGlyph{Tailo}{s}{un}{1}} & {孫} & {\ToSubGlyph{Tailo}{tsh}{ut}{4}} & {出} & {\ToSubGlyph{Tailo}{b}{ong}{2}} & {罔} & {\ToSubGlyph{Tailo}{k}{ok}{4}} & {國} \\
	\hline
	/jɔm/ & iom & /jɔp̚/ & iop & /jun/ & iun & /jut̚/ & iut & /jɔŋ/ & iong & /jɔk̚/ & iok \\
	{\ToSubGlyph{Tailo}{}{iom}{1}} & {} & {\ToSubGlyph{Tailo}{}{iop}{4}} & {} & {\ToSubGlyph{Tailo}{}{iun}{1}} & {} & {\ToSubGlyph{Tailo}{}{iut}{4}} & {} & {\ToSubGlyph{Tailo}{t}{iong}{1}} & {中} & {\ToSubGlyph{Tailo}{kh}{iok}{4}} & {曲} \\
	\hline
	/am/ & am & /ap̚/ & ap & /an/ & an & /aŋ/ & /at̚/ & at & ang & /ak̚/ & ak \\
	{\ToSubGlyph{Tailo}{tsh}{am}{1}} & {參} & {\ToSubGlyph{Tailo}{tsh}{ap}{4}} & {插} & {\ToSubGlyph{Tailo}{s}{an}{1}} & {山} & {\ToSubGlyph{Tailo}{b}{at}{8}} & {密} & {\ToSubGlyph{Tailo}{s}{ang}{1}} & {鬆} & {\ToSubGlyph{Tailo}{p}{ak}{4}} & {北} \\
	\hline
	/jam/ & iam & /jap̚/ & iap & /ɛn/ & ian & /ɛt̚/ & iat & /jaŋ/ & iang & /jak̚/ & iak\\
	{\ToSubGlyph{Tailo}{tshi}{iam}{1}} & {簽} & {\ToSubGlyph{Tailo}{tsh}{iap}{4}} & {接} & {\ToSubGlyph{Tailo}{}{ian}{1}} & {軒} & {\ToSubGlyph{Tailo}{b}{iat}{8}} & {滅} & {\ToSubGlyph{Tailo}{h}{iang}{2}} & {享} & {\ToSubGlyph{Tailo}{kh}{iak}{8}} & {確} \\
	\hline
	/wam/ & uam & /wap̚/ & uap & /wan/ & uan & /wat̚/ & uat & /waŋ/ & uang & /wak̚/ & uak \\
	{\ToSubGlyph{Tailo}{}{uam}{1}} & {} & {\ToSubGlyph{Tailo}{}{uap}{4}} & {} & {\ToSubGlyph{Tailo}{th}{uan}{5}} & {傳} & {\ToSubGlyph{Tailo}{g}{uat}{8}} & {月} & {\ToSubGlyph{Tailo}{tsh}{uang}{2}} & {闖} & {\ToSubGlyph{Tailo}{}{uak}{4}} & {} \\
}%
\MyTable{Rimes without Vowels}{tab:ConRime}{CC|CC|CC|CC}{%
	/m̩/ & m & /m̩ʔ/ & mh & /ŋ̍/ & ng & /ŋ̍ʔ/ & ngh \\
	{\ToSubGlyph{Tailo}{}{m}{7}} & {毋} & {\ToSubGlyph{Tailo}{h}{m}{4}} & {摁} & {\ToSubGlyph{Tailo}{}{ng}{5}} & {黃} & {\ToSubGlyph{Tailo}{h}{ng}{4}} & {哼} \\
}%
}%
\clearpage
{%
\setmainfont{DoulosSIL}%
\setlength{\tabcolsep}{1pt}%
\MyTable{Rimes without Canonical Consonantal Coda}{tab:ColRIme}{CC|CC|Cc|Cc}{
	\multicolumn{2}{|c|}{Unchecked} & \multicolumn{2}{c|}{Checked} & \multicolumn{2}{c|}{Nasalized unchecked} & \multicolumn{2}{c|}{Nasalized checked} \\
	\hline
	{/i/} & {i} & {/iʔ/} & {ih} & {/ĩ/} & {inn} & {/ĩʔ/} & {innh} \\
	{\ToSubGlyph{Tailo}{}{i}{1}} & {伊} & {\ToSubGlyph{Tailo}{p}{i}{4}} & {鱉} & {\ToSubGlyph{Tailo}{}{inn}{5}} & {圓} & {\ToSubGlyph{Tailo}{}{inn}{4}} & {} \\
	\hline
	/u/ & u & /uʔ/ & uh & /ũ/ & unn & /ũʔ/ & unnh \\
	{\ToSubGlyph{Tailo}{b}{u}{2}} & {母} & {\ToSubGlyph{Tailo}{s}{u}{4}} & {欶} & {\ToSubGlyph{Tailo}{}{unn}{5}} & {} & {\ToSubGlyph{Tailo}{}{unnh}{4}} & {} \\
	\hline
	/ju/ & iu & /juʔ/ & iuh & /jũ/ & iunn & /jũʔ/ & iunnh \\
	{\ToSubGlyph{Tailo}{}{iu}{5}} & {油} & {\ToSubGlyph{Tailo}{t}{iu}{4}} & {搐} & {\ToSubGlyph{Tailo}{si}{iunn}{7}} & {想} & {\ToSubGlyph{Tailo}{}{iunn}{4}} & {} \\
	\hline
	/wi/ & ui & /wiʔ/ & uih & /wĩ/ & uinn & /wĩʔ/ & uinnh \\
	{\ToSubGlyph{Tailo}{}{ui}{7}} & {位} & {\ToSubGlyph{Tailo}{h}{ui}{4}} & {血} & {\ToSubGlyph{Tailo}{h}{uinn}{5}} & {橫} & {\ToSubGlyph{Tailo}{}{uinnh}{4}} & {} \\
	\hline
	/e/ & e & /eʔ/ & eh & /ẽ/ & enn & /ẽʔ/ & ennh \\
	{\ToSubGlyph{Tailo}{k}{e}{1}} & {家} & {\ToSubGlyph{Tailo}{b}{eh}{4}} & {欲} & {\ToSubGlyph{Tailo}{kh}{enn}{1}} & {坑} & {\ToSubGlyph{Tailo}{kh}{ennh}{8}} & {喀} \\
	\hline
	/o/ & o & /oʔ/ & oh & /õ/ & onn & /õʔ/ & onnh \\
	{\ToSubGlyph{Tailo}{ph}{o}{1}} & {波} & {\ToSubGlyph{Tailo}{}{o}{8}} & {學}  & {\ToSubGlyph{Tailo}{}{onn}{3}} & {惡} & {\ToSubGlyph{Tailo}{h}{onnh}{4}} & {乎} \\
	\hline
	/jo/ & io & /joʔ/ & ioh & /jõ/ & ionn & /jõʔ/ & ionnh \\
	{\ToSubGlyph{Tailo}{k}{io}{3}} & {叫} & {\ToSubGlyph{Tailo}{tsi}{ioh}{4}} & {借} & {\ToSubGlyph{Tailo}{}{ionn}{1}} & {} & {\ToSubGlyph{Tailo}{}{ionnh}{4}} & {} \\
	\hline
	/we/ & ue & /weʔ/ & ueh & /wẽ/ & uenn & /wẽʔ/ & uennh \\
	{\ToSubGlyph{Tailo}{k}{ue}{2}} & {粿} & {\ToSubGlyph{Tailo}{k}{ue}{4}} & {郭} & {\ToSubGlyph{Tailo}{}{uenn}{1}} & {} & {\ToSubGlyph{Tailo}{}{uennh}{4}} & {} \\
	\hline
	/a/ & a & /aʔ/ & ah & /ã/ & ann & /ãʔ/ & annh \\
	{\ToSubGlyph{Tailo}{ts}{a}{2}} & {早} & {\ToSubGlyph{Tailo}{}{ah}{4}} & {鴨}  & {\ToSubGlyph{Tailo}{s}{ann}{1}} & {衫} & {\ToSubGlyph{Tailo}{s}{annh}{4}} & {煞} \\
	\hline
	/ɔ/ & oo & /ɔʔ/ & ooh & /ɔ̃/ & oonn & /ɔ̃ʔ/ & oonnh \\
	{\ToSubGlyph{Tailo}{g}{oo}{7}} & {五} & {\ToSubGlyph{Tailo}{m}{ooh}{8}} & {膜} & {\ToSubGlyph{Tailo}{}{oonn}{1}} & {} & {\ToSubGlyph{Tailo}{}{oonnh}{4}} & {} \\
	\hline
	/ja/ & ia & /jaʔ/ & iah & /jã/ & iann & /jãʔ/ & iannh \\
	{\ToSubGlyph{Tailo}{si}{ia}{2}} & {寫} & {\ToSubGlyph{Tailo}{tsi}{iah}{8}} & {食}  & {\ToSubGlyph{Tailo}{k}{iann}{1}} & {京} & {\ToSubGlyph{Tailo}{h}{iannh}{4}} & {嚇} \\
	\hline
	/wa/ & ua & /waʔ/ & uah & /wã/ & uann & /wãʔ/ & uannh \\
	{\ToSubGlyph{Tailo}{s}{ua}{1}} & {沙} & {\ToSubGlyph{Tailo}{j}{uah}{8}} & {熱}  & {\ToSubGlyph{Tailo}{}{uann}{7}} & {換} & {\ToSubGlyph{Tailo}{}{uannh}{4}} & {} \\
	\hline
	/ai̯/ & ai & /ai̯ʔ/ & aih & /ãi̯/ & ainn & /ãi̯ʔ/ & ainnh \\
	{\ToSubGlyph{Tailo}{ts}{ai}{1}} & {知} & {\ToSubGlyph{Tailo}{}{aih}{4}} & {哎}  & {\ToSubGlyph{Tailo}{ph}{ainn}{2}} & {歹} & {\ToSubGlyph{Tailo}{}{ainn}{4}} & {} \\
	\hline
	/au̯/ & au & /au̯ʔ/ & auh & /ãu̯/ & aunn & /ãu̯ʔ/ & aunnh \\
	{\ToSubGlyph{Tailo}{ts}{au}{2}} & {走} & {\ToSubGlyph{Tailo}{ph}{auh}{8}} & {雹}  & {\ToSubGlyph{Tailo}{}{aunn}{1}} & {} & {\ToSubGlyph{Tailo}{kh}{aunnh}{8}} & {摳} \\
	\hline
	/jau̯/ & iau & /jau̯ʔ/ & iauh & /jãu̯/ & iaunn & /jãu̯ʔ/ & iaunnh \\
	{\ToSubGlyph{Tailo}{kh}{iau}{2}} & {巧} & {\ToSubGlyph{Tailo}{ng}{iauh}{8}} & {蟯}  & {\ToSubGlyph{Tailo}{}{iaunn}{1}} & {喓} & {\ToSubGlyph{Tailo}{}{iaunnh}{4}} & {} \\
	\hline
	/wai̯/ & uai & /wai̯ʔ/ & uaih & /wãi̯/ & uainn & /wãi̯ʔ/ & uainnh \\
	{\ToSubGlyph{Tailo}{k}{uai}{3}} & {怪} & {\ToSubGlyph{Tailo}{}{uaih}{4}} & {}  & {\ToSubGlyph{Tailo}{s}{uainn}{7}} & {檨} & {\ToSubGlyph{Tailo}{}{uainnh}{4}} & {歪} \\
}%
}%%

\end{document}
