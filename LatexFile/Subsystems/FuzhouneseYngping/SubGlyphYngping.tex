\documentclass[12pt]{article}

% Please typeset with XeLaTeX

\input{../../premable/of\_Subsystems.tex}%

\usepackage{arydshln}% array dash line

% characters with no font supporting
\newcommand{\YngpingCooIQ}{%
	\makebox[1em][c]{\small{扌}\kern-0.6em\small{苴}}%
}%
\newcommand{\YngpingTuhAR}{%
	\makebox[1em][c]{\small{賣}\kern-0.6em\small{刀}}%
}
\newcommand{\YngpingTouhIR}{%
	\makebox[1em][c]{\small{扌}\kern-0.6em\small{禿}}%
}

\begin{document}
% 文章內容

%自定義大標題
\begin{center}
    {\Huge \textbf{榕拼} \par}
    {\Huge \textbf{Fuzhounese Yngping} \par}
    {\Large Hylu \par}
    \hrulefill
\end{center}

This document describes SubGlyph of Yngping, a romanization system of Fuzhounese. I'd like to thank the following websites for their comprehensive information. \par
\begin{itemize}
	\item 榕拼:https://yngping.zingzeu.org/spec/v0.4.0-preview2/main.html
\end{itemize}

\section{Onsets}
{\setmainfont{DoulosSIL}%
\MyTable{Onsets}{tab:Onset}{c|C|C|C|C|C}{%
	Type & P & T & S & C & K \\
	\hline
	全清 & /p/ & /t/ & /t͡s/ & /ȶ͡ɕ/ & /k/ \\
	{} & {\bpmf{\Onset{HagpinXiian}{b}{}} b} & {\bpmf{\Onset{HagpinXiian}{d}{}} d} & {\bpmf{\Onset{HagpinXiian}{z}{}} z} & {\bpmf{\Onset{HagpinXiian}{j}{}} j} & {\bpmf{\Onset{HagpinXiian}{g}{}} g} \\
	\hline
	次清 & /pʰ/ & /tʰ/ & /t͡sʰ/ & /ȶ͡ɕʰ/ & /kʰ/ \\
	{} & {\bpmf{\Onset{HagpinXiian}{p}{}} p} & {\bpmf{\Onset{HagpinXiian}{t}{}} t} & {\bpmf{\Onset{HagpinXiian}{c}{}} c} & {\bpmf{\Onset{HagpinXiian}{q}{}} q} & {\bpmf{\Onset{HagpinXiian}{k}{}} k} \\
	\hline
	次濁 & /m/ & /n/ & {} & {} & /ŋ/ \\
	{} & {\bpmf{\Onset{HagpinXiian}{m}{}} m} & {\bpmf{\Onset{HagpinXiian}{n}{}} n} & {} & {} & {\bpmf{\Onset{HagpinXiian}{ng}{}} ng} \\
	\hline
	清 & /f/ & {} & /s/ & /ɕ/ & /h/ \\
	{} & {\bpmf{\Onset{HagpinXiian}{f}{}} f} & {} & {\bpmf{\Onset{HagpinXiian}{s}{}} s} & {\bpmf{\Onset{HagpinXiian}{x}{}} x} & {\bpmf{\Onset{HagpinXiian}{h}{}} h} \\
	\hline
	濁 & /v/ & /l/ & {} & {} & {} \\
	{} & {\bpmf{\Onset{HagpinXiian}{v}{}} v} & {\bpmf{\Onset{HagpinXiian}{l}{}} l} & {} & {} & {} \\
}%
}%%
\clearpage

\section{Tones}
\MyTable{Tones}{tab:Tone}{c|C|C|C|C|C|C|C|C|C}{%
	Number & 1 & 2 & 3 & 4 & 5 & 6 & 7 & 8 & 9 \\
	\hline
	Code & p,i & s,i & q,i & p,a & s,a & q,a & r,i & r,z & r,a \\
	\hline
	Basic glyph & \ToneCoda{\Tonetype{Jyutping}{1}}{\Toneclass{Jyutping}{1}}{}  & \ToneCoda{\Tonetype{Jyutping}{2}}{\Toneclass{Jyutping}{2}}{} & \ToneCoda{\Tonetype{Jyutping}{3}}{\Toneclass{Jyutping}{3}}{} & \ToneCoda{\Tonetype{Jyutping}{4}}{\Toneclass{Jyutping}{4}}{} & \ToneCoda{\Tonetype{Jyutping}{5}}{\Toneclass{Jyutping}{5}}{} & \ToneCoda{\Tonetype{Jyutping}{6}}{\Toneclass{Jyutping}{6}}{} & \ToneCoda{\Tonetype{Jyutping}{7}}{\Toneclass{Jyutping}{7}}{} & \ToneCoda{\Tonetype{Jyutping}{8}}{\Toneclass{Jyutping}{8}}{} & \ToneCoda{\Tonetype{Jyutping}{9}}{\Toneclass{Jyutping}{9}}{} \\
	\hline
	Labial glyph & \ToneCoda{\Tonetype{Jyutping}{1}}{\Toneclass{Jyutping}{1}}{m}  & \ToneCoda{\Tonetype{Jyutping}{2}}{\Toneclass{Jyutping}{2}}{m} & \ToneCoda{\Tonetype{Jyutping}{3}}{\Toneclass{Jyutping}{3}}{m} & \ToneCoda{\Tonetype{Jyutping}{4}}{\Toneclass{Jyutping}{4}}{m} & \ToneCoda{\Tonetype{Jyutping}{5}}{\Toneclass{Jyutping}{5}}{m} & \ToneCoda{\Tonetype{Jyutping}{6}}{\Toneclass{Jyutping}{6}}{m} & \ToneCoda{\Tonetype{Jyutping}{7}}{\Toneclass{Jyutping}{7}}{m} & \ToneCoda{\Tonetype{Jyutping}{8}}{\Toneclass{Jyutping}{8}}{m} & \ToneCoda{\Tonetype{Jyutping}{9}}{\Toneclass{Jyutping}{9}}{m} \\
	\hline
	Alveolar glyph & \ToneCoda{\Tonetype{Jyutping}{1}}{\Toneclass{Jyutping}{1}}{n}  & \ToneCoda{\Tonetype{Jyutping}{2}}{\Toneclass{Jyutping}{2}}{n} & \ToneCoda{\Tonetype{Jyutping}{3}}{\Toneclass{Jyutping}{3}}{n} & \ToneCoda{\Tonetype{Jyutping}{4}}{\Toneclass{Jyutping}{4}}{n} & \ToneCoda{\Tonetype{Jyutping}{5}}{\Toneclass{Jyutping}{5}}{n} & \ToneCoda{\Tonetype{Jyutping}{6}}{\Toneclass{Jyutping}{6}}{n} & \ToneCoda{\Tonetype{Jyutping}{7}}{\Toneclass{Jyutping}{7}}{n} & \ToneCoda{\Tonetype{Jyutping}{8}}{\Toneclass{Jyutping}{8}}{n} & \ToneCoda{\Tonetype{Jyutping}{9}}{\Toneclass{Jyutping}{9}}{n} \\
	\hline
	Labial glyph & \ToneCoda{\Tonetype{Jyutping}{1}}{\Toneclass{Jyutping}{1}}{ng}  & \ToneCoda{\Tonetype{Jyutping}{2}}{\Toneclass{Jyutping}{2}}{ng} & \ToneCoda{\Tonetype{Jyutping}{3}}{\Toneclass{Jyutping}{3}}{ng} & \ToneCoda{\Tonetype{Jyutping}{4}}{\Toneclass{Jyutping}{4}}{ng} & \ToneCoda{\Tonetype{Jyutping}{5}}{\Toneclass{Jyutping}{5}}{ng} & \ToneCoda{\Tonetype{Jyutping}{6}}{\Toneclass{Jyutping}{6}}{ng} & \ToneCoda{\Tonetype{Jyutping}{7}}{\Toneclass{Jyutping}{7}}{ng} & \ToneCoda{\Tonetype{Jyutping}{8}}{\Toneclass{Jyutping}{8}}{ng} & \ToneCoda{\Tonetype{Jyutping}{9}}{\Toneclass{Jyutping}{9}}{ng} \\
}%%

\section{Rimes}
{\setmainfont{DoulosSIL}%
\setlength{\tabcolsep}{1pt}%
\MyTable{Rimes without Tense/Loose Counterparts}{tab:RimesNoLoose}{CC|CC|CC|CC|CC|CC}{%
	\multicolumn{2}{|c|}{/-∅/} & \multicolumn{2}{c|}{/-ʔ/} & \multicolumn{2}{c|}{/-i̯, -y̆/} & \multicolumn{2}{c|}{/-u̯/} & \multicolumn{2}{c|}{/-ŋ/} & \multicolumn{2}{c|}{/-k̚/} \\
	\hline
	/a/ & a & /aʔ/ & ah & /ai̯/ & ai & /au̯/ & au & /aŋ/ & ang & /ak̚/ & ak \\
	\ToSubGlyph{Yngping}{g}{a}{1} & {家} & \ToSubGlyph{Yngping}{k}{ah}{4} & {客} & \ToSubGlyph{Yngping}{d}{ai}{3} & {帶} & \ToSubGlyph{Yngping}{b}{au}{1} & {包} & \ToSubGlyph{Yngping}{d}{ang}{7} & {鄭} & \ToSubGlyph{Yngping}{k}{ak}{4} & {恰} \\
	\hline
	/ja/ & ia & /jaʔ/ & iah & /jai̯/ & iai & /jau̯/ & iau & /jaŋ/ & iang & /jak̚/ & iak \\
	\ToSubGlyph{Yngping}{c}{ia}{1} & {車} & \ToSubGlyph{Yngping}{b}{iah}{4} & {壁} & \ToSubGlyph{Yngping}{}{iai}{1} & {} & \ToSubGlyph{Yngping}{}{iau}{1} & {} & \ToSubGlyph{Yngping}{g}{iang}{1} & {驚} & \ToSubGlyph{Yngping}{t}{iak}{4} & {眨} \\
	\hline
	/wa/ & ua & /waʔ/ & uah & /wai̯/ & uai & /wau̯/ & uau & /waŋ/ & uang & /wak̚/ & uak \\
	\ToSubGlyph{Yngping}{g}{ua}{1} & {瓜} & \ToSubGlyph{Yngping}{}{uah}{8} & {畫} & \ToSubGlyph{Yngping}{ng}{uai}{2} & {我} & \ToSubGlyph{Yngping}{}{uau}{1} & {} & \ToSubGlyph{Yngping}{}{uang}{2} & {碗} & \ToSubGlyph{Yngping}{}{uak}{8} & {活} \\
	\hline
	/je/ & ie & /jeʔ/ & ieh & \multicolumn{4}{c|}{\MaybeGray{}} & /jeŋ/ & ieng & /jek̚/ & iek \\
	\ToSubGlyph{Yngping}{p}{ie}{1} & {批} & \ToSubGlyph{Yngping}{m}{ieh}{4} & {乜} & \multicolumn{4}{c|}{\MaybeGray{}} & \ToSubGlyph{Yngping}{b}{ieng}{1} & {邊} & \ToSubGlyph{Yngping}{t}{iek}{4} & {鐵} \\
	\hline
	/wo/ & uo & /woʔ/ & uoh & \multicolumn{4}{c|}{\MaybeGray{}} & /woŋ/ & uong & /wok̚/ & uok \\
	\ToSubGlyph{Yngping}{k}{uo}{3} & {課} & \ToSubGlyph{Yngping}{}{uoh}{4} & {沃} & \multicolumn{4}{c|}{\MaybeGray{}} & \ToSubGlyph{Yngping}{h}{uong}{7} & {遠} & \ToSubGlyph{Yngping}{}{uok}{8} & {越} \\
	\hline
	/ɥo/ & yo & /ɥoʔ/ & yoh & \multicolumn{4}{c|}{\MaybeGray{}} & /ɥoŋ/ & yong & /ɥok̚/ & yok \\
	\ToSubGlyph{Yngping}{g}{yo}{5} & {茄} & \ToSubGlyph{Yngping}{}{yoh}{8} & {藥} & \multicolumn{4}{c|}{\MaybeGray{}} & \ToSubGlyph{Yngping}{}{yong}{2} & {養} & \ToSubGlyph{Yngping}{}{yok}{8} & {弱} \\
}
}%%s
\clearpage
I have considered to write loose rimes in the form of its tense counterpart with dakuten. For example, the pronunciation of "骨" might be noted as \SubGlyph{kq}{}{vcou,v}{}{ng}{r}{i}. Despite, Yngping did not distinguish the tokens of being tense or loose - like "ouk" being the loose counterpart of "uk" and tense counterpart of "oouk" simultaneously. Since I am not a Fuzhounese speaker, I decided to assign different symbols rather than directly use dakuten. \par
{\setmainfont{DoulosSIL}%
\setlength{\tabcolsep}{1pt}%
\MyTable{Rimes with Loose Counterparts}{tab:RimesLoose}{c|CC|CC|CC|CC|CC}{%
	T/L & \multicolumn{2}{c|}{/-∅/} & \multicolumn{2}{c|}{/-ʔ/} & \multicolumn{2}{c|}{/-i̯, -y̆, -u̯/} & \multicolumn{2}{c|}{/-ŋ/} & \multicolumn{2}{c|}{/-k̚/} \\
	\hline
	Tense & /o/ & o & /oʔ/ & oh & {} & {} & /oʊ̯ŋ/ & oung & /oʊ̯k̚/ & ouk \\
	{} & \ToSubGlyph{Yngping}{g}{o}{1} & {歌} & \ToSubGlyph{Yngping}{}{oh}{8} & {學} & {} & {} & \ToSubGlyph{Yngping}{g}{oung}{2} & {講} & \ToSubGlyph{Yngping}{g}{ouk}{8} & {滑} \\
	\hdashline
	Loose & /ɔ/ & oo & /ɔʔ/ & ooh & {} & {} & /ɔʊ̯ŋ/ & ooung & /ɔʊ̯k̚/ & oouk \\
	{} & \ToSubGlyph{Yngping}{h}{oo}{7} & {號} & \ToSubGlyph{Yngping}{}{ooh}{4} & {臒} & {} & {} & \ToSubGlyph{Yngping}{k}{ooung}{3} & {困} & \ToSubGlyph{Yngping}{g}{oouk}{4} & {骨} \\
	\hline
	Tense & /ø/ & eo & /øʔ/ & eoh & /øy̆/ & eoy & /øy̆ŋ/ & eoyng & /øy̆k̚/ & eoyk \\
	{} & \ToSubGlyph{Yngping}{c}{eo}{1} & {初} & \ToSubGlyph{Yngping}{g}{eoh}{4} & {嗝} & \ToSubGlyph{Yngping}{d}{eoy}{1} & {堆} & \ToSubGlyph{Yngping}{g}{eoyng}{1} & {江} & \ToSubGlyph{Yngping}{t}{eoyk}{8} & {讀} \\
	\hdashline
	Loose & /ɔ/ & oo & {} & {} & /ɔy̆/ & ooy & /ɔy̆ŋ/ & ooyng & /ɔy̆k̚/ & ooyk \\
	{} & \ToSubGlyph{Yngping}{c}{oo}{3} & {\YngpingCooIQ} & {} & {} & \ToSubGlyph{Yngping}{d}{ooy}{3} & {對} & \ToSubGlyph{Yngping}{g}{ooyng}{7} & {共} & \ToSubGlyph{Yngping}{d}{ooyk}{4} & {觸} \\
	\hline
	Tense & /e/ & e & /eʔ/ & eh & /eu̯/ & eu & /ei̯ŋ/ & eing & /ei̯k̚/ & eik \\
	{} & \ToSubGlyph{Yngping}{s}{e}{1} & {西} & \ToSubGlyph{Yngping}{z}{eh}{8} & {漬} & \ToSubGlyph{Yngping}{h}{eu}{5} & {侯} & \ToSubGlyph{Yngping}{d}{eing}{2} & {點} & \ToSubGlyph{Yngping}{d}{eik}{8} & {特} \\
	\hdashline
	Loose & /a/ & a & {} & {} & /au̯/ & au & /ai̯ŋ/ & aing & /ai̯k̚/ & aik \\
	{} & \ToSubGlyph{Yngping}{s}{a}{3} & {細} & {} & {} & \ToSubGlyph{Yngping}{h}{au}{7} & {候} & \ToSubGlyph{Yngping}{d}{aing}{3} & {店} & \ToSubGlyph{Yngping}{d}{aik}{4} & {得} \\
	\hline
	Tense & /i/ & i & /iʔ/ & ih & /iu̯/ & iu & /iŋ/ & ing & /ik̚/ & ik \\
	{} & \ToSubGlyph{Yngping}{m}{i}{2} & {米} & \ToSubGlyph{Yngping}{d}{ih}{8} & {挃} & \ToSubGlyph{Yngping}{g}{iu}{5} & {球} & \ToSubGlyph{Yngping}{m}{ing}{5} & {閩} & \ToSubGlyph{Yngping}{s}{ik}{8} & {習} \\
	\hdashline
	Loose & /ei̯/ & ei & /ei̯ʔ/ & eih & \multicolumn{2}{c|}{\MaybeGray{}} & /ei̯ŋ/ & eing & /ei̯k̚/ & eik \\
	{} & \ToSubGlyph{Yngping}{m}{ei}{7} & {味} & \ToSubGlyph{Yngping}{c}{eih}{4} & {㲺} & \multicolumn{2}{c|}{\MaybeGray{}} & \ToSubGlyph{Yngping}{m}{eing}{3} & {面} & \ToSubGlyph{Yngping}{s}{eik}{4} & {式} \\
	\hline
	Tense & /u/ & u & /uʔ/ & uh & /ui̯/ & ui & /uŋ/ & ung & /uk̚/ & uk \\
	{} & \ToSubGlyph{Yngping}{}{u}{1} & {烏} & \ToSubGlyph{Yngping}{t}{uh}{8} & {\YngpingTuhAR} & \ToSubGlyph{Yngping}{b}{ui}{1} & {飛} & \ToSubGlyph{Yngping}{z}{ung}{2} & {總} & \ToSubGlyph{Yngping}{d}{uk}{8} & {獨} \\
	\hdashline
	Loose & /oʊ̯/ & ou & /oʊ̯ʔ/ & ouh & \multicolumn{2}{c|}{\MaybeGray{}} & /oʊ̯ŋ/ & oung & /oʊ̯k̚/ & ouk \\
	{} & \ToSubGlyph{Yngping}{}{ou}{7} & {有} & \ToSubGlyph{Yngping}{t}{ouh}{4} & {\YngpingTouhIR} & \multicolumn{2}{c|}{\MaybeGray{}} & \ToSubGlyph{Yngping}{z}{oung}{3} & {俊} & \ToSubGlyph{Yngping}{d}{ouk}{4} & {涿} \\
	\hline
	Tense & /y/ & y & /yʔ/ & yh & \multicolumn{2}{c|}{\MaybeGray{}} & /yŋ/ & yng & /yk̚/ & yk \\
	{} & \ToSubGlyph{Yngping}{d}{y}{1} & {豬} & \ToSubGlyph{Yngping}{}{yh}{8} & {} & \multicolumn{2}{c|}{\MaybeGray{}} & \ToSubGlyph{Yngping}{}{yng}{5} & {榕} & \ToSubGlyph{Yngping}{s}{yk}{8} & {熟} \\
	\hdashline
	Loose & /øy̆/ & eoy & /øy̆ʔ/ & eoyh & \multicolumn{2}{c|}{\MaybeGray{}} & /øy̆ŋ/ & eoyng & /øy̆k̚/ & eoyk \\
	{} & \ToSubGlyph{Yngping}{d}{eoy}{7} & {箸} & \ToSubGlyph{Yngping}{}{eoyh}{4} & {喐} & \multicolumn{2}{c|}{\MaybeGray{}} & \ToSubGlyph{Yngping}{s}{eoyng}{7} & {頌} & \ToSubGlyph{Yngping}{s}{eoyk}{4} & {肅} \\
}
}%

\end{document}