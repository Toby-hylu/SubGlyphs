\documentclass[12pt]{article}

% Please typeset with XeLaTeX

\input{../../premable/of\_Subsystems.tex}

\newcommand{\HakkaKioiIQ}{%
	\makebox[1em][c]{%
		{疒}\kern-0.65em\raisebox{0.1ex}{\scriptsize{彖}}%
	}%
}%

\begin{document}
% 文章內容

%自定義大標題
\begin{center}
    {\Huge \textbf{客家話拼音(四縣腔)} \par}
    {\Huge \textbf{Hakka Pinim (Xi'ian Dialect)} \par}
    {\Large Hylu \par}
    \hrulefill
\end{center}
This document provides SubGlyph of Pinim, a romanization system of Hakka. I'd like to thank the following websites for their comprehensive information. \par
\begin{itemize}
	\item 本土語言資源網:https://mhi.moe.edu.tw
	\item 教育部臺灣客語辭典:https://hakkadict.moe.edu.tw
\end{itemize}

\section{Onset}
{\setmainfont{DoulosSIL}%
\MyTable{Onsets}{tab:Onset}{c|C|C|C|C|C}{%
	Type & P & T & S & C & K \\
	\hline
	全清 & /p/ & /t/ & /t͡s/ & /ȶ͡ɕ/ & /k/ \\
	{} & {\bpmf{\Onset{HagpinXiian}{b}{}} b} & {\bpmf{\Onset{HagpinXiian}{d}{}} d} & {\bpmf{\Onset{HagpinXiian}{z}{}} z} & {\bpmf{\Onset{HagpinXiian}{j}{}} j} & {\bpmf{\Onset{HagpinXiian}{g}{}} g} \\
	\hline
	次清 & /pʰ/ & /tʰ/ & /t͡sʰ/ & /ȶ͡ɕʰ/ & /kʰ/ \\
	{} & {\bpmf{\Onset{HagpinXiian}{p}{}} p} & {\bpmf{\Onset{HagpinXiian}{t}{}} t} & {\bpmf{\Onset{HagpinXiian}{c}{}} c} & {\bpmf{\Onset{HagpinXiian}{q}{}} q} & {\bpmf{\Onset{HagpinXiian}{k}{}} k} \\
	\hline
	次濁 & /m/ & /n/ & {} & {} & /ŋ/ \\
	{} & {\bpmf{\Onset{HagpinXiian}{m}{}} m} & {\bpmf{\Onset{HagpinXiian}{n}{}} n} & {} & {} & {\bpmf{\Onset{HagpinXiian}{ng}{}} ng} \\
	\hline
	清 & /f/ & {} & /s/ & /ɕ/ & /h/ \\
	{} & {\bpmf{\Onset{HagpinXiian}{f}{}} f} & {} & {\bpmf{\Onset{HagpinXiian}{s}{}} s} & {\bpmf{\Onset{HagpinXiian}{x}{}} x} & {\bpmf{\Onset{HagpinXiian}{h}{}} h} \\
	\hline
	濁 & /v/ & /l/ & {} & {} & {} \\
	{} & {\bpmf{\Onset{HagpinXiian}{v}{}} v} & {\bpmf{\Onset{HagpinXiian}{l}{}} l} & {} & {} & {} \\
}%
}%%

\section{Tone}
\MyTable{Tones}{tab:Tone}{c|C|C|C|C|C|C|C|C|C}{%
	Number & 1 & 2 & 3 & 4 & 5 & 6 & 7 & 8 & 9 \\
	\hline
	Code & p,i & s,i & q,i & p,a & s,a & q,a & r,i & r,z & r,a \\
	\hline
	Basic glyph & \ToneCoda{\Tonetype{Jyutping}{1}}{\Toneclass{Jyutping}{1}}{}  & \ToneCoda{\Tonetype{Jyutping}{2}}{\Toneclass{Jyutping}{2}}{} & \ToneCoda{\Tonetype{Jyutping}{3}}{\Toneclass{Jyutping}{3}}{} & \ToneCoda{\Tonetype{Jyutping}{4}}{\Toneclass{Jyutping}{4}}{} & \ToneCoda{\Tonetype{Jyutping}{5}}{\Toneclass{Jyutping}{5}}{} & \ToneCoda{\Tonetype{Jyutping}{6}}{\Toneclass{Jyutping}{6}}{} & \ToneCoda{\Tonetype{Jyutping}{7}}{\Toneclass{Jyutping}{7}}{} & \ToneCoda{\Tonetype{Jyutping}{8}}{\Toneclass{Jyutping}{8}}{} & \ToneCoda{\Tonetype{Jyutping}{9}}{\Toneclass{Jyutping}{9}}{} \\
	\hline
	Labial glyph & \ToneCoda{\Tonetype{Jyutping}{1}}{\Toneclass{Jyutping}{1}}{m}  & \ToneCoda{\Tonetype{Jyutping}{2}}{\Toneclass{Jyutping}{2}}{m} & \ToneCoda{\Tonetype{Jyutping}{3}}{\Toneclass{Jyutping}{3}}{m} & \ToneCoda{\Tonetype{Jyutping}{4}}{\Toneclass{Jyutping}{4}}{m} & \ToneCoda{\Tonetype{Jyutping}{5}}{\Toneclass{Jyutping}{5}}{m} & \ToneCoda{\Tonetype{Jyutping}{6}}{\Toneclass{Jyutping}{6}}{m} & \ToneCoda{\Tonetype{Jyutping}{7}}{\Toneclass{Jyutping}{7}}{m} & \ToneCoda{\Tonetype{Jyutping}{8}}{\Toneclass{Jyutping}{8}}{m} & \ToneCoda{\Tonetype{Jyutping}{9}}{\Toneclass{Jyutping}{9}}{m} \\
	\hline
	Alveolar glyph & \ToneCoda{\Tonetype{Jyutping}{1}}{\Toneclass{Jyutping}{1}}{n}  & \ToneCoda{\Tonetype{Jyutping}{2}}{\Toneclass{Jyutping}{2}}{n} & \ToneCoda{\Tonetype{Jyutping}{3}}{\Toneclass{Jyutping}{3}}{n} & \ToneCoda{\Tonetype{Jyutping}{4}}{\Toneclass{Jyutping}{4}}{n} & \ToneCoda{\Tonetype{Jyutping}{5}}{\Toneclass{Jyutping}{5}}{n} & \ToneCoda{\Tonetype{Jyutping}{6}}{\Toneclass{Jyutping}{6}}{n} & \ToneCoda{\Tonetype{Jyutping}{7}}{\Toneclass{Jyutping}{7}}{n} & \ToneCoda{\Tonetype{Jyutping}{8}}{\Toneclass{Jyutping}{8}}{n} & \ToneCoda{\Tonetype{Jyutping}{9}}{\Toneclass{Jyutping}{9}}{n} \\
	\hline
	Labial glyph & \ToneCoda{\Tonetype{Jyutping}{1}}{\Toneclass{Jyutping}{1}}{ng}  & \ToneCoda{\Tonetype{Jyutping}{2}}{\Toneclass{Jyutping}{2}}{ng} & \ToneCoda{\Tonetype{Jyutping}{3}}{\Toneclass{Jyutping}{3}}{ng} & \ToneCoda{\Tonetype{Jyutping}{4}}{\Toneclass{Jyutping}{4}}{ng} & \ToneCoda{\Tonetype{Jyutping}{5}}{\Toneclass{Jyutping}{5}}{ng} & \ToneCoda{\Tonetype{Jyutping}{6}}{\Toneclass{Jyutping}{6}}{ng} & \ToneCoda{\Tonetype{Jyutping}{7}}{\Toneclass{Jyutping}{7}}{ng} & \ToneCoda{\Tonetype{Jyutping}{8}}{\Toneclass{Jyutping}{8}}{ng} & \ToneCoda{\Tonetype{Jyutping}{9}}{\Toneclass{Jyutping}{9}}{ng} \\
}%%
\clearpage

\section{Rime}
{\setmainfont{DoulosSIL}%
\setlength{\tabcolsep}{1pt}%
\MyTable{Rimes}{tab:Rime}{CC|CC|CC|CC|CC|CC|CC|CC|CC}{%
	\multicolumn{2}{|c|}{/-∅/} & \multicolumn{2}{|c|}{/-i̯/} & \multicolumn{2}{|c|}{/-u̯/} & \multicolumn{2}{|c|}{/-m/} & \multicolumn{2}{|c|}{/-p̚/} & \multicolumn{2}{|c|}{/-n/} & \multicolumn{2}{|c|}{/-t̚/} & \multicolumn{2}{|c|}{/-ŋ/} & \multicolumn{2}{|c|}{/-k̚/} \\
	\hline
	/iː/ & i & \multicolumn{2}{c|}{\MaybeGray{}} & /iːu̯/ & iu & /iːm/ & im & /iːp̚/ & ip & /iːn/ & in & /iːt̚/ & it & /eŋ/ & ing & /ek̚/ & ik \\
	\ToSubGlyph{Jyutping}{c}{i}{1} & 痴 & \multicolumn{2}{c|}{\MaybeGray{}} & \ToSubGlyph{Jyutping}{c}{iu}{1} & 超 & \ToSubGlyph{Jyutping}{c}{im}{1} & 籤 & \ToSubGlyph{Jyutping}{c}{ip}{8} & 妾 &  \ToSubGlyph{Jyutping}{c}{in}{1} & 千 & \ToSubGlyph{Jyutping}{c}{it}{8} & 設 & \ToSubGlyph{Jyutping}{c}{ing}{1} & 青 & \ToSubGlyph{Jyutping}{c}{ik}{7} & 戚 \\
	\hline
	/yː/ & yu & \multicolumn{4}{c|}{\MaybeGray{}} & /yːm/ & yum & /yːp̚/ & yup & /yːn/ & yun & /yːt̚/ & yut & /yŋ/ & yung & /yk̚/ & yuk \\
	\ToSubGlyph{Jyutping}{c}{yu}{1} & 廚 & \multicolumn{4}{c|}{\MaybeGray{}} & \ToSubGlyph{Jyutping}{}{yum}{1} & {} & \ToSubGlyph{Jyutping}{}{yup}{7} & {} &  \ToSubGlyph{Jyutping}{c}{yun}{1} & 川 & \ToSubGlyph{Jyutping}{c}{yut}{8} & 卒 & \ToSubGlyph{Jyutping}{}{yung}{1} & {} & \ToSubGlyph{Jyutping}{}{yuk}{7} & {} \\
	\hline
	/uː/ & u & /uːy̆/ & ui & \multicolumn{2}{c|}{\MaybeGray{}} & /om/ & um & /op̚/ & up & /uːn/ & un & /uːt̚/ & ut & /oŋ/ & ung & /ok̚/ & uk \\
	\ToSubGlyph{Jyutping}{f}{u}{3} & 富 & \ToSubGlyph{Jyutping}{f}{ui}{1} & 灰 & \multicolumn{2}{c|}{\MaybeGray{}} & \ToSubGlyph{Jyutping}{}{um}{1} & {} & \ToSubGlyph{Jyutping}{}{up}{7} & {} &  \ToSubGlyph{Jyutping}{f}{un}{2} & 款 & \ToSubGlyph{Jyutping}{f}{ut}{8} & 闊 & \ToSubGlyph{Jyutping}{f}{ung}{6} & 奉 & \ToSubGlyph{Jyutping}{f}{uk}{9} & 服 \\
	\hline
	/ɛː/ & e & /ei̯/ & ei & /ɛːu̯/ & eu & /ɛːm/ & em & /ɛːp̚/ & ep & /en/ & en & /ɛːt̚/ & et & /ɛːŋ/ & eng & /ɛːk̚/ & ek \\
	\ToSubGlyph{Jyutping}{k}{e}{4} & 伽 & \ToSubGlyph{Jyutping}{k}{ei}{4} & 其 & \ToSubGlyph{Jyutping}{d}{eu}{6} & 掉 & \ToSubGlyph{Jyutping}{l}{em}{2} & 舐 & \ToSubGlyph{Jyutping}{g}{ep}{9} & 夾 &  \ToSubGlyph{Jyutping}{}{en}{1} & {} & \ToSubGlyph{Jyutping}{}{et}{7} & {} & \ToSubGlyph{Jyutping}{d}{eng}{2} & 頂 & \ToSubGlyph{Jyutping}{d}{ek}{7} & 笛 \\
	\hline
	/ɵ/ & eo & /ɵy̆/ & eoi & /ɵu̯/ & eou & /ɵm/ & eom & /ɵp̚/ & eop & /ɵn/ & eon & /ɵt̚/ & eot & /ɵŋ/ & eong & /ɵk̚/ & eok \\
	\ToSubGlyph{Jyutping}{}{eo}{1} & {} & \ToSubGlyph{Jyutping}{g}{eoi}{6} & 巨 & \ToSubGlyph{Jyutping}{}{eou}{1} & {} & \ToSubGlyph{Jyutping}{}{eom}{1} & {} & \ToSubGlyph{Jyutping}{}{eop}{7} & {} &  \ToSubGlyph{Jyutping}{s}{eon}{4} & 屯 & \ToSubGlyph{Jyutping}{c}{eot}{7} & 出 & \ToSubGlyph{Jyutping}{}{eong}{1} & {} & \ToSubGlyph{Jyutping}{}{eok}{7} & {} \\
	\hline
	/œ/ & oe & /œy̆/ & oei & /œu̯/ & oeu & /œm/ & oem & /œp̚/ & oep & /œn/ & oen & /œt̚/ & oet & /œŋ/ & oeng & /œk̚/ & oek \\
	\ToSubGlyph{Jyutping}{d}{oe}{2} & {朵} & \ToSubGlyph{Jyutping}{}{oei}{1} & {} & \ToSubGlyph{Jyutping}{}{oeu}{1} & {} & \ToSubGlyph{Jyutping}{}{oem}{1} & {} & \ToSubGlyph{Jyutping}{}{oep}{7} & {} &  \ToSubGlyph{Jyutping}{}{oen}{1} & {} & \ToSubGlyph{Jyutping}{}{oet}{7} & {} & \ToSubGlyph{Jyutping}{c}{oeng}{4} & {羊} & \ToSubGlyph{Jyutping}{k}{oek}{8} & {卻} \\
	\hline
	/ɔː/ & o & /ɔːi̯/ & oi & /ou̯/ & ou & /ɔːm/ & om & /ɔːp̚/ & op & /ɔːn/ & on & /ɔːt̚/ & ot & /ɔːŋ/ & ong & /ɔːk̚/ & ok \\
	\ToSubGlyph{Jyutping}{c}{o}{5} & {坐} & \ToSubGlyph{Jyutping}{c}{oi}{4} & {才} & \ToSubGlyph{Jyutping}{c}{ou}{4} & {且} & \ToSubGlyph{Jyutping}{}{om}{1} & {} & \ToSubGlyph{Jyutping}{}{op}{7} & {} &  \ToSubGlyph{Jyutping}{}{on}{1} & {安} & \ToSubGlyph{Jyutping}{g}{ot}{8} & {葛} & \ToSubGlyph{Jyutping}{b}{ong}{1} & {邦} & \ToSubGlyph{Jyutping}{ng}{ok}{8} & {惡} \\
	\hline
	/ɐ/ & a & /ɐi̯/ & ai & /ɐu̯/ & au & /ɐm/ & am & /ɐp̚/ & ap & /ɐn/ & an & /ɐt̚/ & at & /ɐŋ/ & ang & /ɐk̚/ & ak \\
	\ToSubGlyph{Jyutping}{}{a}{1} & {} & \ToSubGlyph{Jyutping}{k}{ai}{1} & {溪} & \ToSubGlyph{Jyutping}{d}{au}{3} & {鬥} & \ToSubGlyph{Jyutping}{g}{am}{1} & {今} & \ToSubGlyph{Jyutping}{g}{ap}{8} & {合} &  \ToSubGlyph{Jyutping}{g}{an}{1} & {斤} & \ToSubGlyph{Jyutping}{g}{at}{7} & {吉} & \ToSubGlyph{Jyutping}{g}{ang}{2} & {耿} & \ToSubGlyph{Jyutping}{b}{ak}{7} & {北} \\
	\hline
	/aː/ & aa & /aːi̯/ & aai & /aːu̯/ & aau & /aːm/ & aam & /aːp̚/ & aap & /aːn/ & aan & /aːt̚/ & aat & /aːŋ/ & aang & /aːk̚/ & aak \\
	\ToSubGlyph{Jyutping}{s}{aa}{1} & {沙} & \ToSubGlyph{Jyutping}{b}{aai}{3} & {拜} & \ToSubGlyph{Jyutping}{b}{aau}{1} & {包} & \ToSubGlyph{Jyutping}{s}{aam}{1} & {三} & \ToSubGlyph{Jyutping}{}{aap}{8} & {押} &  \ToSubGlyph{Jyutping}{s}{aan}{2} & {散} & \ToSubGlyph{Jyutping}{f}{aat}{8} & {法} & \ToSubGlyph{Jyutping}{s}{aang}{1} & {生} & \ToSubGlyph{Jyutping}{b}{aak}{9} & {白} \\
	\hline
	\multicolumn{6}{|c|}{\MaybeGray{}} & /m̩/ & m & \multicolumn{6}{c|}{\MaybeGray{}} & /ŋ̍/ & ng & \multicolumn{2}{c|}{\MaybeGray{}} \\
	\multicolumn{6}{|c|}{\MaybeGray{}} & \ToSubGlyph{Jyutping}{}{m}{4} & 唔 & \multicolumn{6}{c|}{\MaybeGray{}} & \ToSubGlyph{Jyutping}{}{ng}{4} & 吳 & \multicolumn{2}{c|}{\MaybeGray{}} \\
}%
}%%

\end{document}