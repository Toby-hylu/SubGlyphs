\section{结语}
本系统比起纯汉字、纯拉丁火汉字拉丁混合排版有以下优势:\par
\begin{itemize}
	\item 排版东亚化:为一个汉字大小的纯粹表音字符,既避免借用同音或谐音汉字的歧义,又保留了东亚文字排版一致性。
	\item 语音模组化:理论上可以依照各种汉语族语言的语音学特性设计出适合的拼合方式。
	\item 跨语言通用:系统保留了不少扩展空间,有潜力作为东亚声调语言的共通表音文字。
\end{itemize}
不过,本系统仅为原型,尚存在以下问题:
\begin{itemize}
	\item 字体无法迁移:由于TikZ与字体交互的特性,目前的优化都是使用TW-Kai字体调整的。改换字体后可能会发生意想不到的偏移错位。
	\item 字体不统一:少数符号TW-Kai显示为明体而非楷体,因此混用了一部分标楷体。字重与基线可能轻微不协调。
	\item 线条风格太细:这是TikZ缩放带来的副作用,若想要克服这个问题,就需要设计专门的字体。作者暂时没有这个时间、精力与能力。
	\item 即使侥幸推广,Unicode支援很可能等不到。以目前Unicode处理拼合文字谚文的策略来说,它傾向於穷举。这个系统能写出的拼合形式显然远超韩文。
\end{itemize}
总而言之,这是我的小小抛砖引玉,期望语言学者、文字设计专家与技术社群能发展出完整、优雅而功能强大的东亚声调语言表音系统,并能达到类似日文那样毫不违和地混合排版的效果,这应该是非常有利于语言保护、传承与白话文学发展的工作。
\clearpage