\section{拼合}
\subsection{基本层}
定义在SGBasic.sty的底层拼合命令为:\par
{\textbackslash}SubGlyph\{声母\}\{介音\}{\{}韵腹{\}}\{韵尾一\}\{韵尾二\}{\{}调型{\}}{\{}调类{\}}。\par
用不到的部分直接在花括号中间留空,以下说明各个部件的格式。\par
\begin{itemize}
	\item 声母:同声母部分
	\item 介音:直接写i、u、y或留空。
	\item 韵腹:同韵母部分。
	\item 扩展韵尾:显示为与韵母拼合的韵尾,一般是offglide形式的i, u, y。
	\item 韵尾二:显示为与声调耦合的韵尾。
	\item 调型、调类:同声调部分。
\end{itemize}
以下是示例表,以声母\Symbol{pq}、韵母\Symbol{oofu}为例。\par
\MyTable{Stacking Example}{tab:Tones}{c |C|C|C|C |C|C|C|C |C|C|C|C}{
	Coda & \multicolumn{4}{c|}{Yin} & \multicolumn{4}{c|}{Yang} & \multicolumn{4}{c|}{Neutral}\\
	\cline{2-13}
	{} & Leveling & Rising & Falling & Entering & Leveling & Rising & Falling & Entering & Leveling & Rising & Falling & Entering \\
	\hline
	Empty & \SubGlyph{pq}{}{oofu}{}{}{p}{i} & \SubGlyph{pq}{}{oofu}{}{}{s}{i} & \SubGlyph{pq}{}{oofu}{}{}{q}{i} & \SubGlyph{pq}{}{oofu}{}{}{r}{i} & \SubGlyph{pq}{}{oofu}{}{}{p}{a} & \SubGlyph{pq}{}{oofu}{}{}{s}{a} & \SubGlyph{pq}{}{oofu}{}{}{q}{a} & \SubGlyph{pq}{}{oofu}{}{}{r}{a} & \SubGlyph{pq}{}{oofu}{}{}{p}{z} & \SubGlyph{pq}{}{oofu}{}{}{s}{z} & \SubGlyph{pq}{}{oofu}{}{}{q}{z} & \SubGlyph{pq}{}{oofu}{}{}{r}{z} \\
	\hline
	Labial & \SubGlyph{pq}{}{oofu}{}{m}{p}{i} & \SubGlyph{pq}{}{oofu}{}{m}{s}{i} & \SubGlyph{pq}{}{oofu}{}{m}{q}{i} & \SubGlyph{pq}{}{oofu}{}{m}{r}{i} & \SubGlyph{pq}{}{oofu}{}{m}{p}{a} & \SubGlyph{pq}{}{oofu}{}{m}{s}{a} & \SubGlyph{pq}{}{oofu}{}{m}{q}{a} & \SubGlyph{pq}{}{oofu}{}{m}{r}{a} & \SubGlyph{pq}{}{oofu}{}{m}{p}{z} & \SubGlyph{pq}{}{oofu}{}{m}{s}{z} & \SubGlyph{pq}{}{oofu}{}{m}{q}{z} & \SubGlyph{pq}{}{oofu}{}{m}{r}{z} \\
	\hline
	Coronal & \SubGlyph{pq}{}{oofu}{}{n}{p}{i} & \SubGlyph{pq}{}{oofu}{}{n}{s}{i} & \SubGlyph{pq}{}{oofu}{}{n}{q}{i} & \SubGlyph{pq}{}{oofu}{}{n}{r}{i} & \SubGlyph{pq}{}{oofu}{}{n}{p}{a} & \SubGlyph{pq}{}{oofu}{}{n}{s}{a} & \SubGlyph{pq}{}{oofu}{}{n}{q}{a} & \SubGlyph{pq}{}{oofu}{}{n}{r}{a} & \SubGlyph{pq}{}{oofu}{}{n}{p}{z} & \SubGlyph{pq}{}{oofu}{}{n}{s}{z} & \SubGlyph{pq}{}{oofu}{}{n}{q}{z} & \SubGlyph{pq}{}{oofu}{}{n}{r}{z} \\
	\hline
	Dorsal & \SubGlyph{pq}{}{oofu}{}{ng}{p}{i} & \SubGlyph{pq}{}{oofu}{}{ng}{s}{i} & \SubGlyph{pq}{}{oofu}{}{ng}{q}{i} & \SubGlyph{pq}{}{oofu}{}{ng}{r}{i} & \SubGlyph{pq}{}{oofu}{}{ng}{p}{a} & \SubGlyph{pq}{}{oofu}{}{ng}{s}{a} & \SubGlyph{pq}{}{oofu}{}{ng}{q}{a} & \SubGlyph{pq}{}{oofu}{}{ng}{r}{a} & \SubGlyph{pq}{}{oofu}{}{ng}{p}{z} & \SubGlyph{pq}{}{oofu}{}{ng}{s}{z} & \SubGlyph{pq}{}{oofu}{}{ng}{q}{z} & \SubGlyph{pq}{}{oofu}{}{ng}{r}{z} \\
	\hline
	Nasalized & \SubGlyph{pq}{}{oofu}{}{hn}{p}{i} & \SubGlyph{pq}{}{oofu}{}{hn}{s}{i} & \SubGlyph{pq}{}{oofu}{}{hn}{q}{i} & \SubGlyph{pq}{}{oofu}{}{hn}{r}{i} & \SubGlyph{pq}{}{oofu}{}{hn}{p}{a} & \SubGlyph{pq}{}{oofu}{}{hn}{s}{a} & \SubGlyph{pq}{}{oofu}{}{hn}{q}{a} & \SubGlyph{pq}{}{oofu}{}{hn}{r}{a} & \SubGlyph{pq}{}{oofu}{}{hn}{p}{z} & \SubGlyph{pq}{}{oofu}{}{hn}{s}{z} & \SubGlyph{pq}{}{oofu}{}{hn}{q}{z} & \SubGlyph{pq}{}{oofu}{}{hn}{r}{z} \\
}%%

\subsection{映射层}
底层命令过于繁琐。例如,汉语拼音的「xióng」,需要历经如下推导过程:
\begin{itemize}
	\item x对应声母「\bpmf{cf}」,其主码为cf;
	\item iong是一个不透明拼式,等效于「ü+eng」,应拆解为「\bpmf{y}」与「\bpmf{nceng}」,其主码分别为y与nceng;
	\item 普通话的二声对应阳平,其调型码为p,调类码为a;
	\item 根据以上信息,调用代码为\detokenize{\SubGlyph{cf}{y}{nceng}{}{}{p}{a}},效果为\SubGlyph{cf}{y}{nceng}{}{}{p}{a}。
\end{itemize}
有鉴于此,package内提供了六种语言的范例及接口sty档案SubGlyph.sty。只需要了解拼式的声韵调分別是哪些,即可自动产生对应的通用拼合式注音,例如:\detokenize{\ToSubGlyph{Hanpin}{x}{iong}{2}}直接得到\ToSubGlyph{Hanpin}{x}{iong}{2}。\par
以下是六种语言的拼字示例。如有疏漏,敬请母语者不吝赐教。\par
\MyTable{Example: Pronunciation of Numbers}{tab:Supported}{c|C|C|C|C|C|C|C|C|C|C|C}{%
	Language & Zero & One & Two & Three & Four & Five & Six & Seven & Eight & Nine & Ten \\
	\hline
	Mandarin &%
		\ToSubGlyph{Hanpin}{l}{ing}{2} &%
		\ToSubGlyph{Hanpin}{}{yi}{1} &%
		\ToSubGlyph{Hanpin}{}{er}{4} &%
		\ToSubGlyph{Hanpin}{s}{an}{1} &%
		\ToSubGlyph{Hanpin}{s}{i}{4} &%
		\ToSubGlyph{Hanpin}{}{wu}{3} &%
		\ToSubGlyph{Hanpin}{l}{iu}{4} &%
		\ToSubGlyph{Hanpin}{q}{i}{1} &%
		\ToSubGlyph{Hanpin}{b}{a}{1} &%
		\ToSubGlyph{Hanpin}{j}{iu}{3} &%
		\ToSubGlyph{Hanpin}{sh}{i}{2} \\%
	\hline
	Taiwanese Colloquial &%
		\ToSubGlyph{Tailo}{l}{ing}{5} &%
		\ToSubGlyph{Tailo}{ts}{it}{8} &%
		\ToSubGlyph{Tailo}{j}{i}{7} &%
		\ToSubGlyph{Tailo}{s}{ann}{1} &%
		\ToSubGlyph{Tailo}{s}{i}{3} &%
		\ToSubGlyph{Tailo}{g}{oo}{7} &%
		\ToSubGlyph{Tailo}{l}{ak}{8} &%
		\ToSubGlyph{Tailo}{tsh}{it}{4} &%
		\ToSubGlyph{Tailo}{p}{eh}{4} &%
		\ToSubGlyph{Tailo}{k}{au}{2} &%
		\ToSubGlyph{Tailo}{ts}{ap}{8} \\%
	\cline{1-1}\cline{3-3}\cline{5-8}\cline{10-12}
	Taiwanese Literary &%
		{} &%
		\ToSubGlyph{Tailo}{}{it}{4} &%
		{} &%
		\ToSubGlyph{Tailo}{s}{am}{1} &%
		\ToSubGlyph{Tailo}{s}{u}{3} &%
		\ToSubGlyph{Tailo}{ng}{oo}{2} &%
		\ToSubGlyph{Tailo}{l}{iok}{8} &%
		{} &%
		\ToSubGlyph{Tailo}{p}{at}{4} &%
		\ToSubGlyph{Tailo}{k}{iu}{2} &%
		\ToSubGlyph{Tailo}{s}{ip}{8} \\%
	\hline
	Hakka (Xiian) &%
		\ToSubGlyph{HagpinXiian}{l}{ang}{2} &%
		\ToSubGlyph{HagpinXiian}{}{id}{5} &%
		\ToSubGlyph{HagpinXiian}{ng}{i}{4} &%
		\ToSubGlyph{HagpinXiian}{s}{am}{1} &%
		\ToSubGlyph{HagpinXiian}{x}{i}{4} &%
		\ToSubGlyph{HagpinXiian}{}{ng}{3} &%
		\ToSubGlyph{HagpinXiian}{l}{iug}{5} &%
		\ToSubGlyph{HagpinXiian}{q}{id}{5} &%
		\ToSubGlyph{HagpinXiian}{b}{ad}{5} &%
		\ToSubGlyph{HagpinXiian}{g}{iu}{3} &%
		\ToSubGlyph{HagpinXiian}{s}{iib}{6} \\%
	\hline
	Cantonese &%
		\ToSubGlyph{Jyutping}{l}{ing}{4} &%
		\ToSubGlyph{Jyutping}{j}{at}{7} &%
		\ToSubGlyph{Jyutping}{j}{i}{6} &%
		\ToSubGlyph{Jyutping}{s}{aam}{1} &%
		\ToSubGlyph{Jyutping}{s}{ei}{3} &%
		\ToSubGlyph{Jyutping}{}{ng}{5} &%
		\ToSubGlyph{Jyutping}{l}{uk}{9} &%
		\ToSubGlyph{Jyutping}{c}{at}{7} &%
		\ToSubGlyph{Jyutping}{b}{aat}{8} &%
		\ToSubGlyph{Jyutping}{g}{au}{2} &%
		\ToSubGlyph{Jyutping}{s}{ap}{9} \\%
	\hline
	Suzhounese &%
		\ToSubGlyph{WuphinSoutseu}{l}{in}{2} &%
		\ToSubGlyph{WuphinSoutseu}{}{iq}{7} &%
		\ToSubGlyph{WuphinSoutseu}{gn}{i}{6} &%
		\ToSubGlyph{WuphinSoutseu}{s}{e}{1} &%
		\ToSubGlyph{WuphinSoutseu}{s}{y}{5} &%
		\ToSubGlyph{WuphinSoutseu}{}{ng}{6} &%
		\ToSubGlyph{WuphinSoutseu}{l}{oq}{8} &%
		\ToSubGlyph{WuphinSoutseu}{tsh}{iq}{7} &%
		\ToSubGlyph{WuphinSoutseu}{p}{oq}{7} &%
		\ToSubGlyph{WuphinSoutseu}{c}{ieu}{3} &%
		\ToSubGlyph{WuphinSoutseu}{z}{eq}{8} \\%
	\hline
	Fuzhounese &%55=1, 33=2, 213=3, 24=4, 53=5, 242=7, 5=8
		\ToSubGlyph{Yngping}{l}{ing}{5} &%
		\ToSubGlyph{Yngping}{s}{uoh}{8} &%文讀eik4?
		\ToSubGlyph{Yngping}{n}{ei}{7} &%
		\ToSubGlyph{Yngping}{s}{ang}{1} &%
		\ToSubGlyph{Yngping}{s}{ei}{3} &%
		\ToSubGlyph{Yngping}{ng}{ou}{7} &%文讀ngu2?
		\ToSubGlyph{Yngping}{l}{eoyk}{8} &%文讀lyk8?
		\ToSubGlyph{Yngping}{c}{eik}{4} &%
		\ToSubGlyph{Yngping}{b}{aik}{4} &%
		\ToSubGlyph{Yngping}{g}{au}{2} &%文讀giu2?
		\ToSubGlyph{Yngping}{s}{eik}{8} \\%
}%%
作者也提供了python脚本,只需要整理该语言的拉丁化方案并参考写好的档案撰写配置档,即可自动产生对应的sty档案。脚本会自动更新顶层的接口档案SubGlyph.sty,不需要手动新增\detokenize{\usepackage{SGYoursystem.sty}}。为了防止撞名,建议命名方式为:\par
\begin{itemize}
	\item 若该拼音方案标准化程度高,请直接使用该语言的简称加上该语言「拼」或「罗」的读音,例如「Jyutping(粵拼)」、「Tailo(台罗)」、「Hanpin(汉拼)」。
	\item 若该拼音方案属于一个大类的子方案,请在后面加上地域名称在该方案中的拼式,例如「WuphinSoutseu(吴拼苏州)」、「HagpinXiian(客拼四县)」。普通话音节组合少,拉丁化后极易重名,所以若您的语言拉丁化方案与汉语拼音有所差异,请直接使用该方案书写地域名称。
\end{itemize}

\subsection{混合排版示例}
\begin{itemize}
	\item 普通话:「和」读音异常复杂,有\ToSubGlyph{Hanpin}{h}{an}{4}、\ToSubGlyph{Hanpin}{h}{e}{2}、\ToSubGlyph{Hanpin}{h}{uo}{4}、\ToSubGlyph{Hanpin}{h}{u}{2}、\ToSubGlyph{Hanpin}{h}{e}{4}等读音。
	\item 台语:伊佇\ToSubGlyph{Tailo}{h}{ia}{1}毋知企偌久矣。
\end{itemize}

\clearpage