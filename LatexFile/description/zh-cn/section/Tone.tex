\section{聲調}
调用声调符号的宏为{\textbackslash}ToneCoda\{调型码\}\{调类码\}\{韵尾码\},定义于SGBasic.sty內。声调标识遵循以下标准:\par
\begin{itemize}
	\item 调型码:平、上、去、入分别对应p, s, q, r。
	\item 调类码:陰、陽、中分别对应i, a, z,中调只是表明其属于扩展调型,并不表示其无法明定阴阳调型,例如粤语之「下阴入」调。
	\item 韵尾码:m, n, ng可以应用于平上去入四个调型,在遇到入声时会自动转为p, t, k形式;p, t, k韵尾只能应用于入声调,应用于其他调型会导致解析失败;若有需求,可以使用对应辅音或元音的码产生扩展标调,例如以下范例提供了填入hn产生鼻化音的四声。
\end{itemize}
\MyTable{声调韵尾组合表}{tab:ToneCoda}{c|C|C|C|C}{
	韵尾{\textbackslash}声调 & 平 & 上 & 去 & 入 \\
	\hline
	无 & \ToneCoda{p}{i}{} & \ToneCoda{s}{i}{} & \ToneCoda{q}{i}{} & \ToneCoda{r}{i}{} \\
	唇 & \ToneCoda{p}{i}{m} & \ToneCoda{s}{i}{m} & \ToneCoda{q}{i}{m} & \ToneCoda{r}{i}{m} \\
	舌尖 & \ToneCoda{p}{i}{n} & \ToneCoda{s}{i}{n} & \ToneCoda{q}{i}{n} & \ToneCoda{r}{i}{n} \\
	舌根 & \ToneCoda{p}{i}{ng} & \ToneCoda{s}{i}{ng} & \ToneCoda{q}{i}{ng} & \ToneCoda{r}{i}{ng} \\
	鼻化 & \ToneCoda{p}{i}{hn} & \ToneCoda{s}{i}{hn} & \ToneCoda{q}{i}{hn} & \ToneCoda{r}{i}{hn} \\
}
\clearpage