\section*{前言}
汉语族语言(Sinitic languages)虽种类繁多,然迄今唯有普通話具备官方地位与成熟、普及的书面语体系。其他语言如台语、客语、粤语等,虽具有准官方地位,并各自发展出一套约定俗成的书写标准,然而仍然面临一些问题。\par
采用“纯汉字书写”时,遇到的主要问题是“有音无字”的处理困境:\par
\begin{itemize}
	\item 造新字:粤语白话文倾向创造新字(如“嚟”),通常以口字旁加声旁构形,虽然较能准确表音但字形可能复杂;
	\item 用普通话音近字:台语白话文则常写作字形简单、音近的普通话汉字(如“袂”),唯这一做法致使“不表音、不表意”的现象普遍存在;
	\item 用同音字:台语白话文也经常使用,如「鲁」肉饭,但失去表意能力,例如曾经有外国人将其翻译为“Shandong meat rice”(误以为“鲁”是山东的意思);
	\item 用本字:看似是最佳方案,但首先台语、粤语乃至吴语、闽东语等都存在底层非汉语词,其次即使有本字也可能生僻难解,如表野草莓之“藨”字不认识者十之八九。
\end{itemize}
采用“纯拉丁书写”时,遇到的主要是方案设计困境:\par
\begin{itemize}
	\item 语言差异:拉丁字母擅长表达的音节与汉语族差异太大,这导致拉丁化方案一般都存在复杂的二合、三合乃至多合字母。例如,表达「咀嚼脆的食物」的拟声词以台罗方案书写是难以认读的“kha̍unnh”,包括二合字母kh、四合韵母aunn、入声韵尾h等部分,而这个字往往以叠词的形式出现。
	\item 连字符:连读变调丰富的语言需要明确区分变调单位,这导致书写时引入大量的连字符,可读性与美观大打折扣。还是以上一个例子,该拟声词往往是叠词的形式“kha̍unnh-kha̍unnh”。
	\item 方案冲突:成熟的拼音方案往往有极大的影响力,迫使设计其他语言的方案时不得不配合已经成熟的方案,例如一些中国民间的拼音方案不得不使用bb或bh表达“浊双唇塞音”,因为b已经被汉语拼音用来表达“清不送气双唇塞音”了。
\end{itemize}

汉字与罗马字混写虽然看似综合了上述两者的好处,但它在美学上缺点极其显著:
\begin{itemize}
	\item 看起来极度不成熟,像是小学生遇到不会写的字,就写成拼音代替那样。
	\item 排版美感糟糕:拉丁文字与东亚文字各自发展出了成熟的排版美学,但这两种美学本质上有冲突:拉丁文字让每个字符有不同的宽度使得书面流畅灵动,东亚文字保证每个字符等宽使得书面整齐干净。混合排版时就会发生严重的不协调感。理工科书籍或许可以接受,但文学类书籍的排版美学是其重要的组成部分,不成熟的排版美学会限制其书面文学的发展。
\end{itemize}

既然混排看起来是有潜力的方向,我们可从已有成熟东亚文字混排历史的日文与韩文的书写策略获得启发:
\begin{itemize}
	\item 日文的策略在于:能书写则用汉字,无合适汉字则改用假名标音,既保留语意,也兼顾语音,实用而灵活。
	\item 韩文的策略则体现在:设计出谚文体系,将音节结构压缩于“一个汉字大小”的方格中,实现书写单元的简洁与规律。
\end{itemize}

有鉴于此,本文尝试提出一种结合两者优点的书写设计:以注音符号为基础,并参考韩文的音节压缩方式,将单一音节(含扩展符号)封装为一个方块单位,进而配合类似日文的汉字-假名混排机制,达成语音准确与排版美观的平衡。\par

在技术层面上,为避免新增 Unicode 码位,本文提出的扩展策略采用既有字符结合方式表现浊音与鼻化:
\begin{itemize}
	\item 使用浊点(゛)与半浊点(゜)扩展符号表;
	\item 适当引入一些笔画简单的汉字或片假名字符;
\end{itemize}
目前,为兼容所有必要字符,本文使用支援完整注音符号集及片假名的字型TW-98楷体(及其增补字体)与标楷体。展望未來,若本系统获得应用与推广,致使Unicode接纳新注音符号字符,则期逐步汰换借用的日文假名与简单汉字,构建风格更加纯粹的注音合字系統。\par
\clearpage