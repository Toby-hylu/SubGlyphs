\section{韻母}
\subsection{單元音}
調用元音符號的宏為{\textbackslash}bpmf\{主碼,附碼\},定義於SGBasic.sty。\par
\begin{itemize}
	\item 主碼:開口度(兩個字母)、舌位(一個字母)、圓唇度(一個字母)。例如,「閉前展唇元音(\bpmf{ccfu})」為{\textbackslash}bpmf\{ccfu\}。如果四字母主碼不夠用,填入阿拉伯數字1~9可得到楷體蘇州碼子作為臨時符號。
	\item 附碼:與輔音一致。
\end{itemize}
方音符號裡表示鼻化韻的符號,在這裡拿來表示非鼻音的單元音。這是因為鼻化既有形式韻尾符號,又可以用附碼,無需單獨符號。鑑於羅馬字系統不區分普通輔音與輔音成音節的拼式,輔音成音節在本系統使用普通輔音注音符號。\par
\begin{figure}[h!]
	\centering
	\begin{tikzpicture}[
		node distance=1.6cm, % 節點間距,可調整
		state/.style={ % 定義節點樣式
			%draw, % 繪製邊框
			%circle, % 圓形
			minimum height=0.4cm, % 最小尺寸
			minimum width=3cm, 
			align=center, % 文字居中
			thick, % 邊框粗細
			font=\bfseries, % 字體加粗
			scale=0.7,
		},
		transition/.style={ % 定義轉換箭頭樣式
			-{Stealth[scale=1.2]}, % 現代箭頭樣式
			thick, % 線條粗細
			shorten >=2pt, % 縮短箭頭,避免碰到圓圈邊緣
			shorten <=2pt % 縮短尾部,避免碰到圓圈邊緣
		},
		condition/.style={ % 定義條件標籤樣式
			font=\small, % 字體較小
			align=center % 文本居中
		}
	]
		%back vowels
		\node[state] (CloseBack) {/ɯ/ {\bpmf{ccbu}} - /u/ {\bpmf{ccbr}}}; 
		\node[state, below=0.4cm of CloseBack] (SemiCloseBack){};
		\node[state, below=0.4cm of SemiCloseBack] (CloseMidBack) {/ɤ/ {\bpmf{cmbu}} - /o/ {\bpmf{cmbr}}}; 
		\node[state, below=0.4cm of CloseMidBack] (MidBack){};
		\node[state, below=0.4cm of MidBack] (OpenMidBack) {/ʌ/ {\bpmf{ombu}} - /ɔ/ {\bpmf{ombr}}};
		\node[state, below=0.4cm of OpenMidBack] (SemiOpenBack){};
		\node[state, below=0.8cm of SemiOpenBack] (OpenBack) {/ɑ/   - /ɒ/  };
		
		\draw (CloseBack) -- (CloseMidBack);
		\draw (CloseMidBack) -- (OpenMidBack);
		\draw (OpenMidBack) -- (OpenBack);
	
		%central vowels
		\node[state, left=of CloseBack] (CloseCentral) {/ɨ/ {\bpmf{cccu}} - /ʉ/ {\bpmf{cccr}}};
		\node[state, left=1.5cm of SemiCloseBack] (SemiCloseCentral){};
		\node[state, left=1.4cm of CloseMidBack] (CloseMidCentral) {/ɘ/   - /ɵ/ {\bpmf{cmcr}}};
		\node[state, left=1.3cm of MidBack] (MidCentral) {/ə/ {\bpmf{mmcu}}};
		\node[state, left=1.2cm of OpenMidBack] (OpenMidCentral) {/ɜ/   - /ɞ/  };
		\node[state, left=1.1cm of SemiOpenBack] (SemiOpenCentral) {/ɐ/ {\bpmf{oscu}}};
		\node[state, left=1.0cm of OpenBack] (OpenCentral) {/ä/   - /ɶ̈/  };
		
		\draw (CloseBack) -- (CloseCentral);
		\draw (CloseMidBack) -- (CloseMidCentral);
		\draw (OpenMidBack) -- (OpenMidCentral);
		\draw (OpenBack) -- (OpenCentral);
		\draw (CloseCentral) -- (CloseMidCentral);
		\draw (CloseMidCentral) -- (MidCentral);
		\draw (MidCentral) -- (OpenMidCentral);
		\draw (OpenMidCentral) -- (SemiOpenCentral);
		\draw (SemiOpenCentral) -- (OpenCentral);
		
		%front vowels
		\node[state, left=of CloseCentral] (CloseFront) {/i/ {\bpmf{ccfu}} - /y/ {\bpmf{ccfr}}};
		\node[state, left=1.4cm of SemiCloseCentral] (SemiCloseFront) {};
		\node[state, left=1.2cm of CloseMidCentral] (CloseMidFront) {/e/ {\bpmf{cmfu}} - /ø/ {\bpmf{cmfr}}};
		\node[state, left=1.0cm of MidCentral] (MidFront) {};
		\node[state, left=0.8cm of OpenMidCentral] (OpenMidFront) {/ɛ/ {\bpmf{omfu}} - /œ/ {\bpmf{omfr}}};
		\node[state, left=0.6cm of SemiOpenCentral] (SemiOpenFront) {};
		\node[state, left=0.4cm of OpenCentral] (OpenFront) {/a/ {\bpmf{oofu}} - /ɶ/  };
		
		\draw (CloseCentral) -- (CloseFront);
		\draw (CloseMidCentral) -- (CloseMidFront);
		\draw (OpenMidCentral) -- (OpenMidFront);
		\draw (OpenCentral) -- (OpenFront);
		\draw (CloseFront) -- (CloseMidFront);
		\draw (CloseMidFront) -- (OpenMidFront);
		\draw (OpenMidFront) -- (OpenFront);
		
		% retroflex vowels
		\node[state, left=of CloseFront] (CloseRetroflex) {/ɿ,ʅ/ {\bpmf{ccxu}} - /ʮ,ʯ/ {\bpmf{ccxr}}};
		\node[state, left=1.4cm of SemiCloseFront] (SemiCloseRetroflex) {};
		\node[state, left=1.2cm of CloseMidFront] (CloseMidRetroflex) {/ᶒ/  }; 
		\node[state, left=1.0cm of MidFront] (MidRetroflex) {/ᶕ/ {\bpmf{mmxu}}};
		\node[state, left=0.8cm of OpenMidFront] (OpenMidRetroflex) {/ᶓ/  };
		\node[state, left=0.6cm of SemiOpenFront] (SemiOpenRetroflex) {};
		\node[state, left=0.4cm of OpenFront] (OpenRetroflex) {/ᶏ/  };
		
		\draw (CloseFront) -- (CloseRetroflex);
		\draw (CloseMidFront) -- (CloseMidRetroflex); 
		\draw (OpenMidFront) -- (OpenMidRetroflex);
		\draw (OpenFront) -- (OpenRetroflex); 
		\draw (CloseRetroflex) -- (CloseMidRetroflex);
		\draw (CloseMidRetroflex) -- (MidRetroflex);
		\draw (MidRetroflex) -- (OpenMidRetroflex);
		\draw (OpenMidRetroflex) -- (OpenRetroflex);
		
		%labels
		\node[state, above=0.4cm of CloseFront] (LabelFront) {f};
		\node[state, above=0.4cm of CloseCentral] (LabelCentral) {c};
		\node[state, above=0.4cm of CloseBack] (LabelBack) {b};
		\node[state, above=0.4cm of CloseRetroflex] (LabelRetroflex) {x};
		\node[state, right=0.2cm of CloseBack] (LabelClose) {cc};
		\node[state, right=0.2cm of CloseMidBack] (LabelCloseMid) {cm};
		\node[state, right=0.2cm of MidBack] (LabelMid) {m};
		\node[state, right=0.2cm of OpenMidBack] (LabelOpenMid) {om};
		\node[state, right=0.2cm of SemiOpenBack] (LabelSemiOpen) {os};
		\node[state, right=0.2cm of OpenBack] (LabelOpen) {oo};
	
	\end{tikzpicture}
	\caption{單元音舌位圖}
	\label{fig:SingleVowels}
\end{figure}

\subsection{雙元音}
原本的鼻化韻符號也改為非鼻化的其他預組元音。\par
\MyTable{預組元音}{tab:Precombined}{c|c|C|C|C|C|C|C}{
	字串開頭 & 韻尾|韻腹 & {\bpmf{oofu}}a & {\bpmf{omfu}}e & {\bpmf{cmbr}}o & {\bpmf{ombr}}oo & {\bpmf{ccfu}}i & {\bpmf{ccbr}}u \\
	\hline
	vc & {\bpmf{ccfu}}i & {\bpmf{vcai}} & {\bpmf{vcei}} & {\bpmf{vcoi}} & {\bpmf{vcooi}} & \MaybeGray{} & {\bpmf{vcui}} \\
	vc & {\bpmf{ccbr}}u & {\bpmf{vcau}} & {\bpmf{vceu}} & {\bpmf{vcou}} & {\bpmf{vcoou}} & {\bpmf{vciu}} & \MaybeGray{} \\
	\hline
	nc & {\bpmf{pn}}m & {\bpmf{ncam}} & {} & {\bpmf{ncom}} & {} & {} & {} \\
	nc & {\bpmf{tn}}n & {\bpmf{ncan}} & {\bpmf{ncen}} & {} & {} & {} & {} \\
	nc & {\bpmf{kn}}ng & {\bpmf{ncang}} & {\bpmf{nceng}} & {\bpmf{ncong}} & {} & {} & {} \\
}%

\clearpage