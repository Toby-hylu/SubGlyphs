\section{結語}
本系統比起純漢字、純拉丁或漢字拉丁混合排版有以下優勢:\par
\begin{itemize}
	\item 排版東亞化:為一個漢字大小的純粹表音字符,既避免借用同音或諧音漢字的歧義,又保留了東亞文字排版一致性。
	\item 語音模組化:理論上可以依照各種漢語族語言的語音學特性設計出適合的拼合方式。
	\item 跨語言通用:系統保留了不少擴展空間,有潛力作為東亞聲調語言的共通表音文字。
\end{itemize}
不過,本系統僅為原型,尚存在以下問題:
\begin{itemize}
	\item 字型無法遷移:由於TikZ與字型交互的特性,目前的最佳化都是使用TW-Kai調整的。改換字型後可能會發生意想不到的偏移錯位。
	\item 字型不統一:少數符號TW-Kai顯示為明體而非楷體,因此混用了一部分標楷體。字重與基線可能輕微不協調。
	\item 線條風格太細:這是TikZ縮放帶來的副作用,若想要克服這個問題,就需要設計專門的字型。作者暫時沒有這個時間、精力與能力。
	\item 即使僥倖推廣,Unicode支援很可能等不到。以目前Unicode處理拼合文字諺文的策略來說,它傾向於枚舉。這個系統能寫出的拼合形式顯然遠超韓文。
\end{itemize}
總而言之,這是我的小小拋磚引玉,期望語言學者、文字設計專家與技術社群能發展出完整、優雅而功能強大的東亞聲調語言表音系統,並能達到類似日文那樣毫不違和地混合排版的效果,這應該是非常有利於語言保護、傳承與白話文學發展的工作。
\clearpage