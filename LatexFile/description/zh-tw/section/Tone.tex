\section{聲調}
調用聲調符號的宏為{\textbackslash}ToneCoda\{調型碼\}\{調類碼\}\{韻尾碼\},定義於SGBasic.sty內。聲調標識遵循以下標準:\par
\begin{itemize}
	\item 調型碼:平、上、去、入分別對應p, s, q, r。
	\item 調類碼:陰、陽、中分別對應i, a, z,中調只是表明其屬於擴展調型,並不表示其無法明定陰陽調型,例如粵語之「下陰入」調。
	\item 韻尾碼:m, n, ng可以應用於平上去入四個調型,在遇到入聲時會自動轉為p, t, k形式;p, t, k韻尾只能應用於入聲調,應用於其他調型會導致解析失敗;若有需求,可以使用對應輔音或元音的碼產生擴展標調,例如以下範例提供了填入hn產生鼻化音的四聲。
\end{itemize}
\MyTable{聲調韻尾組合表}{tab:ToneCoda}{c|C|C|C|C}{
	韻尾{\textbackslash}聲調 & 平 & 上 & 去 & 入 \\
	\hline
	無 & \ToneCoda{p}{i}{} & \ToneCoda{s}{i}{} & \ToneCoda{q}{i}{} & \ToneCoda{r}{i}{} \\
	唇 & \ToneCoda{p}{i}{m} & \ToneCoda{s}{i}{m} & \ToneCoda{q}{i}{m} & \ToneCoda{r}{i}{m} \\
	舌尖 & \ToneCoda{p}{i}{n} & \ToneCoda{s}{i}{n} & \ToneCoda{q}{i}{n} & \ToneCoda{r}{i}{n} \\
	舌根 & \ToneCoda{p}{i}{ng} & \ToneCoda{s}{i}{ng} & \ToneCoda{q}{i}{ng} & \ToneCoda{r}{i}{ng} \\
	鼻化 & \ToneCoda{p}{i}{hn} & \ToneCoda{s}{i}{hn} & \ToneCoda{q}{i}{hn} & \ToneCoda{r}{i}{hn} \\
}
\clearpage