\section{Conclusion}
The proposed system has advantages comparing to pure Han character, pure latin letter or Han-Latin mixed writing systems: \par
\begin{itemize}
	\item East Asian typeset style: Phonograms with the size of one Han character, unified typeset style while avoiding the misunderstanding. 
	\item Modularized phones: in theory, one may design a matched system for each East Asian tonal languages. 
	\item Extension potential: might be used as an unified phonogram system for East Asian languages. 
\end{itemize}
Despite, this is a prototype. It still has problems below: 
\begin{itemize}
	\item Font unchangeable: The current optimization is based on the display of TW-Kai-98 with TikZ. If the font changes, misplacement may happen. 
	\item Font ununified: some symbols are not displayed correctly in TW-Kai as Kai, for which some characters are assigned as Biau-Kai. Baselines and weights may not fit perfectly. 
	\item Line style: A result of TikZ scaling. The only way to overcome it is to design a new font, which is beyond my capability. 
	\item May not be supported by Unicode: Unicode pre-combined 11172 Hanguls. But this system has even more possible combinations than Hangul. 
\end{itemize}
In all, this is my prototype. I hope linguistic experts, symbol designers and tech groups may develop complete, elegant and powerful system for East Asian tonal languages and achieve the performance of Japanese mixed typesetting. This will highly benefit the language protection, inheritance and literal development. 
\clearpage